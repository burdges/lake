\documentclass[twoside,letterpaper]{sig-alternate}
\usepackage[margin=1in]{geometry}
\usepackage[utf8]{inputenc}
\usepackage{url}
\usepackage{eurosym}
\usepackage{tikz}
%\usepackage{listings}
%\usepackage{graphicx}
%\usepackage{wrapfig}
%\usepackage{caption}
%\usepackage{subcaption}

\usetikzlibrary{cd}
%\usetikzlibrary{shapes,arrows}
%\usetikzlibrary{positioning}
%\usetikzlibrary{calc}

\def\mathcomma{,}
\def\mathperiod{.}

\def\mathcomma{}
\def\mathperiod{}

\title{Identity-based encryption for meta-data protection when first establishing contacts}
% \subtitle{}
\author{Jeffrey Burdges, Florian Dold, Christian Grothoff, Neal Wa{\bf r}field}
\date{\today}

\begin{document}
\maketitle


\begin{abstract}
We argue that a limited use of identity-base encryption would improve
the security and confidentiality of some user-friendly end-to-end
encrypted messaging protocols.
\end{abstract}


\section{Introduction}

% \subsection{Setting}

A secure end-to-end encrypted channel requires that endpoints
authenticate one another, usually by verifing the fingerprint of
one another's public key material.
Assistance in verifying this material therefore becomes an important
feature. % of encryption protocols.

At the same time, the mechanism for obtaining key material tend to
leak meta-data about user's social graph, which proves harmful to 
...~\cite{skynet,??metadatapolution??}.

Available decentralized solutions like the Web-of-Trust~\cite{wot}
leak social graph data since the key signing graph is public and
key lookups use a handfull of key servers.  GNS provides a model
resembling the web-of-trust that leaks less profusely, but offers
no protection against confirmation attacks~\cite{gns}.  
Both these schemes face deployment and adoption challenges.

Instead, there is wave of newer more user-friendly encrypted messaging
applications that distribute public key information using a centralized
trusted authorities, or sometimes trust-on-first-use (TOFU). 

In Signal~\cite{TextSecure} for example, there is a central key server
from which clients obtain a pre-key that consists of a long-term key
and a preliminary Axolotl ratchet state.  
This long term key provides a fingerprint that users could verify
out-of-band, or verbally, for authentication. 

There are two fundemental problems with using centralized key servers
in this way.  
First, users tend not to actually verify fingerprints, so 
key servers could be coerced into substituting a false key.
Signal itself actively encourages users to authenticate fingerprints,
but proprietary efforts tend not to show fingerprints by default.
Second, the authority learns who resolved which names.

Signal's Axolotl ratchet has impressive forward-secrecy properties
that strengthen any protocol parts that depend upon TOFU, and
reduces the sheer diversity of attacks on systems like PGP.
Of course, forward-secrecy cannot replace authentication.
Yet, Sgnal provides deniability for example.

\smallskip
% \subsection{Goals}

We propose that, systems like Signal that already rely on an authority,
should avoid exposing meta data to the authority by using
identity-based encryption (IBE) when first establishing a contact.
In this first message, we share the long-term fingerprint key and
ratchet state, which must themselves be real non-identity-based key
material. 

In essence, we employ IBE only for a weak form of authentication, 
while full authentication still requires that users verify fingerprints
TOFU style.
In doing so, we prevent meta data leakage to the authority, and reduce
its attack surface for manipulating pre-keys.  

Our scheme remains vulnerable to an attacker who can both
 extract the recipient's private key from the authority, and
 attack the transport during the initial exchange.  
In particular, we loose if the authority's servers provide the transport
layer as well, which costs us our hard won anoymity anyways. 

We need two features to address this :

First, we need an anonymity preserving transport layer where recipients
are determined using an identity.  We could for example adapt Tor to
support an identity-based naming scheme for hidden services.  

Second, we should decentralize our IBE scheme's private key issuing
authority by allowing contact initiators to expect the responder
obtain a private key from several authorites as in~\cite{}. 

Alone neither appears sufficent, but taken together they provide 
considerably stronger protections. 


\section{Ratchet spinup protocol}

\def\E{\mathbb{E}}
\def\F{\mathbb{F}}
\def\Z{\mathbb{Z}}
\def\ID{\mathtt{ID}}
\def\rk{\mathtt{rk}}
\def\ck{\mathtt{ck}}
\def\mk{\mathtt{mk}}

\subsection{Identity-based scheme}

Along with a long-term fingerprint key,
a Signal pre-key contains the initial elliptic curve point required
by the Axolotl ratchet, meaning
 the first message can operate the ratchet as usual.
We shall instead derive the initial root key from a identity-base key
exchange, which we prefer over a public key system,
 like Sakai–Kasahara scheme (SAKKE),
for simplicity when integrating with the ratchet.
Indeed, we even choose to employ the chain key machinery of Axolotl,
 in case we need to send multiple messages. 

We must however choose a key exchange that does not leak the recipent's
identity, which many IBE schemes do~\cite{AnonIBE}.  
We find one such scheme implicit in the Boneh–Franklin IBE scheme 
\cite{BF-IBE} (see \cite[??]{BoyenMIBS}).

In the Boneh–Franklin IBE setting,
we have an elliptic curve $E_1$ over a field $\F_1$,
 along with an element $P_1 \in E_1$ of order $q$.
In addition, we have a group $E_2$ of order $q$ and 
 a pairing $e : E_1 \times E1 \to \mathbb{E}_2$.

We employ the usual setup phase: 
Our private key issing authority
 selects a private key $s \in \F^\times$,
 derives the public key $K = s P_1$, and
 selects two public hash functions
  $H_1 : \{0,1\}^* \to \E_1$ and $H_1 : E_2 \to \{0,1\}^n \to \E_1$.

We use usual extraction phase too: 
Our private key issing authority
issues the private key $d_\ID = s H_1(\ID)$ to anyone who
 can prove possesion of the identity $\ID$.

\subsection{Axolotl tweaks} % motivation/outline
% \medskip

We first outline our tweaks to the Axolotl ratchet before 
 spelling out the protocol in more detail.

As a rule, pairing friendly curve are large, slow, and relatively
 weak cryptographically.
It follows that, after this initial contact request, we should 
transition the ratchet to a faster higher security curve $\mathbb{E}$,
such as curve25519 \cite{DJB-Curve25519}
 or Ed448-Goldilocks \cite{Ed448-Goldilocks}.
Let $P$ denote the distinguished generator of $\mathbb{E}$.

We cannot transform the initiator's key pair for
 our pairing friendly curve system
into a key pair on $\mathbb{E}$ \cite{??no_homomorphism??}.
As a result, any message sent on this initial Axolotl chain must include
both our first Axolotl public key $B = b P \in \mathbb{E}$, along with
 our long-term fingerprint key $F \in \mathbb{E}$.

We mentioned above using the chain key machinery of Axolotl even from
the first message.  There are risks in this decission because the
authority can decrypt these messages.  It appears sufficent that
contact request messages either disallow user content all togehter,
perhaps a desirable feature for preventing SPAM in any case, or that 
perhaps users merely be warned not to speak freely.

As usual in Axolotl, we must hold fixed the initiator's public side of
the identity-base key exchange throughout the lifetime of this initial
chain key.  Interestingly, we need not necessarily retain the 
initiator's private key for the identity-base key exchange, and
 the responder need only run the key exchange once as well.
We do retain the private key $b$ for $B$ until the responder uses it.

As a final tweak to Axolotl, we should incorporate the long-term
fingerprint key by using triple Diffie-Hellman during early rounds.
We still lack confidence in a user's fingerprint until
we have seen it used however.
We could speed this up by using two long-term fingerprint keys,
an initiator one in the pairing system, and
 a responder key for $\mathbb{E}$,
with the fingerprint being the hash of both.

\subsection{Axolotl round} % one
% \medskip

An initiator knows their responder's identity $\ID_R$, so they may
compute the responder's public key as $g_{\ID_R} = e(H_1(\ID_R),K)$.
To setup a contact, the initiator first chooses a random $r \in \Z/q\Z$
to compute the initial root key $\rk = H_2( r g_{\ID_R} )$,
chain key $\ck_I = H(\mathrm{``Chain''} || \rk)$, and
ephemeral public key $U = r P$.
In addition, they choose a scalar $b$ for $\E$,
 and then save $(\rk,\ck_I,U,b)$.

An initial message should communicate $(B,F_I,\ID_I)$ where 
 $B = b P$ is the initiator's first regular Axolotl public-key,
 $F_I$ is the initiator's long-term fingerprint key, and
 $\ID_I$ is the initiators identity.
So, to send the contact request, the initiator
sends $U$ along with $E_{\mk}(B,F_I,\ID_I,\ldots)$ where 
 $\mk = H(\mathrm{``Message''}  || \ck_I)$ is the message key,
and replaces the saved $\ck_I$ with $H(\mathrm{``Iterate''} || \ck_I)$.

Of course, this sending process may be repreated with possibly different
values for $\ldots$, but they remain only as secure as $d_{\ID_R}$,
making them vulnerable to the authority.

% \subsection{Axolotl round two}
\smallskip

To decode the contact request message, the respondent 
first computes the initial root key $\rk = H_2( e(d_{\ID_R}, U) )$
and their incoming chain key $\ck_I = H(\mathrm{``Chain''} || \rk)$.
These shared secrets agree because 
$$ e(d_{\ID_R}, U) = e(s H_1(\ID_R), r P) = r e(H_1(\ID_R), s P) = r g_{\ID_R} $$
Next they compute the initiator's
 message key $\mk = H(\mathrm{``Message''}  || \ck_I)$ 
to attempt to decrypt the message body.
If this succeeds, the respondent learns and saves $(B,F_I,\ID_I,\ldots)$
 under the index $H(U)$.
If not, they may save the skipped message keys $\mk$ under
the index $H(U)$, and iterate $\ck_I$, as usual in Axolotl.

To accept the contact request,
the respondend first chooses a random $t \in \Z/q\Z$ and 
 computes $T = H_2( t e(H_1(\ID_I),d_{\ID_R}) )$ and $V = t H(\ID_R)$.
Along with this, they choose a random scalar $a$ for $\E$ to
iterate to the second root key $\rk := H(\rk, T, a B, a F_I)$ and
 computes their outgoing chain key $\ck_R = H(\mathrm{``Chain''} || \rk)$,
and then saves $(\rk,\ck_I,\ck_R,V,B,a)$. 

An accept message should communicate $(A,F_R)$ where 
 $A = a P$ is the respondent's first regular Axolotl public-key, and
 $F_I$ is the respondent's long-term fingerprint key.
So, to send the acceptance, the respondent
sends $(H(B),V)$ along with $E_{\mk}(A,F_R,\ldots)$ where again
 $\mk = H(\mathrm{``Message''}  || \ck_R)$ is the message key,
and replaces the saved $\ck_R$ with $H(\mathrm{``Iterate''} || \ck_R)$.
We could optionally some header encryption to $V$, but this seems unecessary.

% \subsection{Axolotl round three}
\smallskip

An initiator processes the acceptance by recognizing $H(B)$.
At which point, they compute $T = H_2( e(d_{\ID_R}, V) )$,
iterate to the second root key $\rk := H(\rk, T, b A, f_I A )$, and 
compute the incoming chain key $\ck_R = H(\mathrm{``Chain''} || \rk)$.
Again, these shared secrets agree because 
$$ t e(H_1(\ID_I), d_{\ID_R}) = t e(H_1(\ID_I), s H_1(\ID_R))
  = e(s H_1(\ID_I), t H_1(\ID_R)) = e(d_{\ID_I}, V) \mathperiod $$
% FIXME: Do we really need $V = t H(\ID_R)$ here?
Next they compute the respondent's
 message key $\mk = H(\mathrm{``Message''}  || \ck_R)$ 
to attempt to decrypt the message body.
If this succeeds, the initiator learns and saves $(A,F_R,\ldots)$.
If not, they may save the skipped message keys $\mk$ under the index $H(U)$,,
and iterate $\ck_R$, as usual in Axolotl.

At this point, we have exausted our identity-base key excahnge scheme,
 as both sides were hashed into the the root key.  In addition, we have
hashed in one exchange $a B = b A$ between strong ephemeral keys, and
one exchange $a F_I = f_I A$ between an ephemeral and
 the long-term fingerprint key $F_I = f_I P$.
It follows that both parties know one another's finger prints, but only
the initiator knows that the respondent saw the correct fingerprint key.
To address this with fewer round trips, we ask that both parties
now fast-forward the root key $\rk$ by replacing it with
 $\rk' := H(\rk f_R B) = H(\rk, b F_R )$.
Both parties may do so as soon as they have computed the respondent's
 first chain key $\ck_R$.
We thus allow Axolotl to run as usual from this point forward, while
ensureing that the initiator's next message conveys that the initator
 has the correct long-term fingerprint key.


\section{Identity-based mixnet} % transport

... describe Tor hidden services with identity-based keys ...


\section{Analysis}

We observe that one principle weakness of our proposal is that pre-keys
offer a degree of ephemerality lacking in identity-based keys.

If for example a phone is stolen, then a Signal user can simply ask
for a new SIM card from their telephone provider, and install Signal
on their new phone.  On install, Signal replaces all their pre-keys 
stored on the key server, invalidating the old ones, and restarts all
the ratchet states with their contacts, thereby making the old phone's
data worthless for impersonation. 

In our identity-based scheme, the old phone retains the victims
identity-based private key $d_{\ID}$, which the thief could exploit for
either impersonation or eves dropping on initial contact request messsages.
We could mitigate theft risks by rotating the authority's private key $s$,
perhaps weekly, but this attack window remains larger than with Signal.

In this vein, an attacker who infiltrates the phone system could steal
the identity-based private key $d_{\ID}$ anytime they like.  
Any such attacker could replace pre-keys and restart ratchets as well,
but this appears tougher to exploit.
% HOUSTON WE HAVE A PROBLEM 

\smallskip


\smallskip

...

There is an interesting unknown key share attack on Signal that
even verfying fingerprints cannot address~\cite[\S4.2]{TextSecure}.
At least one proposal for fixing the Unknown Key Share attack
modifies the pre-keys to leak not only all new contacts, but
all currently known possible contacts~\cite[\S4.3]{TextSecure}! 
% FIXME Answer: Read \S4.2 and \S4.3 in TextSecure if the above makes no sense.
As with many pre-key designs, this fix may damage the deniability
 of an early message by incorprotating a signature.

We avert the unknown key share attack under the assumption that
an attacker cannot learn private IBE keys from the authority, and
 without causing a toxic metadata spill.
We also avoid sacrifice deniability during these initial couple steps of
the ratchet by instead incorporating key exchanges with long-term
fingerprint keys, much like triple Diffie-Hellman does.
We pay a price in delaying certainty of the fingerprint, but
 that is unavoidable in our context.


Alternatively, one could employ triple Diffie-Hellman inside the
Axolotl ratchet during two initial steps,
 thus sacrificing deniability during those steps,
 and delaying certainty of the fingerprint. 



\section{Decentralization}

Signal could reduce the vulnerability of pre-keys by using
multiple pre-keys distributed to multiple organizations,
 possibly in different legal environment, but
doing so risks exposing user's meta-data to more parties.

In our identity-base scheme, there is no such risk from using
multiple key generation authorities in different jurisdictions.

... describe or cite multi-authority IBE ...


\section{Conclusions}

...


% \section*{Acknowledgements}
% This work benefits from the financial support of the Brittany Region
% (ARED 9178) and a grant from the Renewable Freedom Foundation.


%\newpage

\bibliographystyle{abbrv}
\bibliography{msg,ecc,ibe}

\end{document}


\section{}
