\documentclass[twoside,letterpaper]{sig-alternate}
\usepackage[margin=1in]{geometry}
\usepackage[utf8]{inputenc}
\usepackage{url}
\usepackage{eurosym}
\usepackage{tikz}
%\usepackage{listings}
%\usepackage{graphicx}
%\usepackage{wrapfig}
%\usepackage{caption}
%\usepackage{subcaption}

\usetikzlibrary{cd}
%\usetikzlibrary{shapes,arrows}
%\usetikzlibrary{positioning}
%\usetikzlibrary{calc}

\def\mathcomma{,}
\def\mathperiod{.}

\def\mathcomma{}
\def\mathperiod{}


\title{Stop Meta Data Leaks with Identity-based Encryption!} % first contact
% \subtitle{}
\author{Jeffrey Burdges, Florian Dold, Christian Grothoff, Neal Wa{\bf r}field}
\date{\today}

\begin{document}
\maketitle

% \section{}

\begin{abstract}
We argue that a limited use of identity-base encryption would improve
the security and confidentiality of certain ``usable'' end-to-end
encrypted messaging protocols.
\end{abstract}

\section{Introduction}

Before secure end-to-end communication channels can be established,
the endpoints need to obtain each other's public keys.  This step is
crucial, as anyone who can observe the key lookup can also tell who is
going to communicate with whom, and designs leaking the social graph
in this manner indirectly support democide~\cite{skynet}.

Even decentralised key management solutions like the
Web-of-Trust~\cite{wot} leak social graph data as the key signing
graph is public and key lookups in the Web-of-Trust go to a few
servers that can hence learn quite a bit about the underlying
communication patterns.  While decentralised solutions that avoid meta
data leakage exist~\cite{gns}, they remain largely experimental at
this time.

In practice, most users either rely on trust-on-first-use (TOFU) or an
{\em authority} provided by a trusted third party to obtain public key
information.  For example, in Signal~\cite{TextSecure}, there is an
central key distribution server from which clients obtain a pre-key
that consists of a long-term key and a preliminary ratchet state.
This may generally suffice to protect the content of one's private
communication, as out-of-band or voice verification can be used to
authenticate the fingerprint provided by the authority.
However, the authority still learns who is resolving which
name.

There are subtleties to the pre-key that mitigate the worst attacks,
like by providing pre-keys for all potential
contacts~\cite[\S4.3]{TextSecure}, but doing do exposes even more
meta-data.
% FIXME: I don't understand. Elaborate: ``all potential''? How does that work?
% I don't see the relationship to 4.3 in the paper cited, that paper
% just seems to fix a minor vulnerability in the KX between the two
% parties.
% FIXME: overall, do we need the above paragraph in the intro!?


For systems that rely on a authority or TOFU for key
exchange, we propose to avoid exposing meta data to the authority
using identity-based encryption (IBE) to lookup a TOFU
signing key.  The TOFU signing key would only be used on first
contact, and is afterwards be replaced by an endpoint-generated public
key (and a ratchet in Axolotl-like systems).

In doing so, we prevent meta data leakage to the authority, limiting
its attack surface to manipulating the TOFU signing key.  As a result,
the combined scheme leaks less meta data and provides strictly
stronger key security than just TOFU or just IBE.

The scheme remains vulnerable to an attacker that can both extract the
recipient's private key from the authority, and perform a
man-in-the-middle attack against the user's transport during the
initial exchange.  In particular, we loose if the authority's servers
provide the transport layer as well.  One could address this using a
secure transport layer, say by adapting Tor hidden services to support
an identity-based scheme for the naming of hidden services.  This
would be ideal, as the use of Tor would then prevent meta-data leakage
for the end-to-end encrypted communication.


\section{Key Exchange}

Along with a long-term fingerprint key,
a Signal pre-key contains the initial elliptic curve point required
by the Axolotl ratchet, meaning
 the first message can operate the ratchet as usual.
%
For us, this first message use a pairing based crypto system
 such as ... \cite{}.
We derive the initial root key from the identity-base key
exchange and even employ the chain key machinery of Axolotl,
 in case we need to send multiple messages.

As a rule, pairing friendly curve are large, slow, and relatively
 weak cryptographically.
We should therefore transition the ratchet to a faster higher security
curve $\mathbb{E}$, such as curve25519 \cite{} or
 Ed448-Goldilocks \cite{}, after the initial contact request.
We cannot transform the initiator's key pair from
 our pairing friendly curve system
into a key pair on $\mathbb{E}$ \cite{??no_homomorphism??}.
As a result, any message sent on this initial chain must include both
 our first Axolotl public key $B \in \mathbb{E}$, along with
 our long-term fingerprint key $F \in \mathbb{E}$.

As usual in Axolotl, we must hold fixed the initiator's public side
 for the identity-base key exchange throughout this initial chain.
In this vein, we need not retain the initiator's private key $u$ for
 the identity-base key exchange, and
 the responder need only run the key exchange once as well.
We do retain the private key $b$ for $B$ until the responder uses it.

As a final tweak to Axolotl, we should incorporate
the long-term fingerprint key by using triple Diffie-Hellman
in the first rounds with $\mathbb{E}$.
We would however lack confidence in a user's fingerprint until
we have seen it used.
We could speed this up by using two long-term fingerprint keys,
an initiator one in the pairing system, and
 a responder key for $\mathbb{E}$,
with the fingerprint being the hash of both.

\section{Transport}

... describe Tor hidden services with identity-based keys ...

\section{Decentralization}

Signal could reduce the vulnerability of pre-keys by using
multiple pre-keys distributed to multiple organizations,
 possibly in different legal environment, but
doing so risks exposing user's meta-data to more parties.

In our identity-base scheme, there is no such risk from using
multiple key generation authorities in different jurisdictions.

... describe or cite multi-authority IBE ...

\section{Conclusions}

...


% \section*{Acknowledgements}
% This work benefits from the financial support of the Brittany Region
% (ARED 9178) and a grant from the Renewable Freedom Foundation.


%\newpage

\bibliographystyle{abbrv}
\bibliography{msg}

\end{document}


\section{}
