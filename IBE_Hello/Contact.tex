\documentclass[twoside,letterpaper]{sig-alternate}
\usepackage[margin=1in]{geometry}
\usepackage[utf8]{inputenc}
\usepackage{url}
\usepackage{eurosym}
\usepackage{tikz}
%\usepackage{listings}
%\usepackage{graphicx}
%\usepackage{wrapfig}
%\usepackage{caption}
%\usepackage{subcaption}

\usetikzlibrary{cd}
%\usetikzlibrary{shapes,arrows}
%\usetikzlibrary{positioning}
%\usetikzlibrary{calc}

\def\mathcomma{,}
\def\mathperiod{.}

\def\mathcomma{}
\def\mathperiod{}

\title{Compairing sets of contacts}
% \subtitle{}
% \author{Jeffrey Burdges, Florian Dold, Christian Grothoff}
\date{\today}

\begin{document}
\maketitle


% \begin{abstract}
% \end{abstract}


\def\E{\mathbb{E}}
\def\F{\mathbb{F}}
\def\Z{\mathbb{Z}}


% \section{Introduction}

% \section{}

\subsection{BLS Signatures}

We first outline the Boneh-Lynn-Shacham (BLS) signature scheme \cite{BLS-SigWeilPairing}.

We have an elliptic curve $E_1$ over a field $\F_1$,
 along with an element $P_1 \in E_1$ of order $q$.
In addition, we have a group $E_2$ of order $q$ and
 a pairing $e : E_1 \times \E_1 \to \E_2$.
% FIXME: describe pairing property 

A private key consists of a scalar $c \in \F_1$, while
the corresponding public key is $C = c P_1$.

We have a hash function $H_1 : \{0,1\}^* \to \E_1$ along with
 our pairing friendly curve setup mentioned above.
In the BLS scheme,
a signature on a message $m$ by $C$ is $\sigma = H(m)^c$.

We verify the signature $\sigma$ by checking that
  $e(C,H(m)) = e(P_1,\sigma)$.
If $\sigma = H(m)^c$, then this holds by the pairing property.

\subsection{Our Protocol}

Initially, any participant $i$ has a public key pair $C_i = c_i P_1$
 for our pairing based signature scheme.  
In addition, any participant $i$ has a contact list $L_i$ consisting
of tuples $(j,C_j,\sigma_{i,j})$ where
 $\sigma_{i,j} = H(i,\texttt{date})^{c_j}$
 is a BLS signature proving that $j$ knows $i$.

At the begining of the contact comparison operation,
our participant $i$ creates an ephemeral private key $t_i \in \F$.
From this, they define 
$$ X_i := % \texttt{sort}
  \left[ (t_i C, t_i \sigma_{i,j}) | (j,C_j,\sigma_{i,j}) \in L_i \right]
  \mathperiod $$ 
If a participant $i$ has obtained the set $X_{i'}$ from
 another participant $i'$, then they define a set 
$$ Y_i := \texttt{sort} \left[ t_i C' | (C',\cdot) \in X_{i'} \right]
  \mathperiod $$ 

Imagine a user Alie wishes to know how many contacts she has in commone
another user Bob.   
First, Alice creates $t_{\textrm{Alice}}$ and
 sends $X_{\textrm{Alice}}$ to Bob.
Next, Bob creates $t_{\textrm{Bob}}$ and
 sends both $X_{\textrm{Bob}}$ and $Y_{\textrm{Bob}}$ back to Alice.
Finally, Alice verifies the signatures in $X_{\textrm{Bob}}$,
 computes $Y_{\textrm{Alice}}$, and find the size of the intersection 
 $I = Y_{\textrm{Alice}} \cap Y_{\textrm{Bob}}$.
If Bob wishes to know the size of the intersection too, then Alice 
can send him $Y_{\textrm{Alice}}$ and he can run the final step as well.

Any common $C \in L_{\textrm{Alice}} \cap L_{\textrm{Bob}}$
 contributes to $I$ because
 $t_{\textrm{Alice}} (t_{\textrm{Bob}} C) = t_{\textrm{Bob}} (t_{\textrm{Alice}} C)$.
Also, the signatures in $X_{\textrm{Bob}}$ validate because 
we employ the BLS pairing based sgnature scheme. 
\begin{align*}
e( t_{\textrm{Bob}} C_j, H(m) )
 = t_{\textrm{Bob}} e( C_j, H(m) ) \\
 \quad = t_{\textrm{Bob}} e(P_1,\sigma_{i,j})
 = e(P_1,t_{\textrm{Bob}} \sigma_{i,j})
\end{align*}




%\newpage

\bibliographystyle{abbrv}
\bibliography{msg,ecc,pairing}

\end{document}



