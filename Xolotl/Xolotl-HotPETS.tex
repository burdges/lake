

\documentclass[twoside,letterpaper]{sig-alternate}
\usepackage[margin=1in]{geometry}
\usepackage[utf8]{inputenc}
\usepackage{url}
\usepackage{eurosym}
\usepackage{tikz}
%\usepackage{listings}
%\usepackage{graphicx}
%\usepackage{wrapfig}
%\usepackage{caption}
%\usepackage{subcaption}

\usetikzlibrary{cd}
%\usetikzlibrary{shapes,arrows}
%\usetikzlibrary{positioning}
%\usetikzlibrary{calc}

\def\mathcomma{,}
\def\mathperiod{.}

\def\mathcomma{}
\def\mathperiod{}


\title{Xolotl ratchet}
\subtitle{A selectively stateful mixnet format for post-quantum security}
\author{Jeffrey Burdges}
\date{\today}

\begin{document}
\maketitle

% \section{}

% L\'aszl\'o Baba's quasi-polynomial time algorithm for graph isomorphism\cite{Babai-GI}

We describe a new double ratchet construction Xolotl,
 inspired by the Axolotl ratchet, % \cite{Axolotl},
that integrates with the Sphinx mix network packet format \cite{Sphinx}.
We argue this opens the door to compact mix network formats with
truly hybrid anonymity, meaning they rests upon the stronger of the
security assumptions required by the different public key primitives employed.


\section{Problem} % Post-quantum key sizes

Amongst the chief obstacles to deploying post-quantum cryptography are
the comparatively large key sizes.  As a comparison : 
%
A recent Ring-LWE key exchange New Hope \cite[\S7, p.10]{NewHope} needs
 key sizes of 1824 or 2048 bytes, both of which must be ephemeral,
while a current McElise-like system McBits % \cite{McBits,InitRec}
 needs a staggering 1MB public keys.
%
Super-singular isogeny Diffie-Hellman (SIDH) \cite[p. 21]{SIDH-2016} keys
are only 564 bytes, or 751 bytes uncompressed, but
 the key exchange requires at least 100 times as much CPU time as
 an ECDH key exchange with equivalent classical security.

Anonymity tools like mix networks are sensitive to key size because 
users interact with numerous nodes and key material overhead is 
quadratic in the number of hops. % $n(n+1)/2$

\smallskip
% \section{Sphinx key blinding}

Sphinx \cite{Sphinx} is a packet format for anonymizing mix networks
that is provably secure in the universal composability framework, and
 addresses the key material burden by mutating or rebinding a
 single ephemeral public key $\alpha$ with each hop,
 as opposed to unwrapping an unrelated public key for each hop.

An ECC key is blinded by multiplication with a shared secret scalar
derived from the Diffie-Hellman exchange:
After selecting an initial private scalar $x_0$,
 public curve point $\alpha_0 = x_0 G$, and 
 a sequence of $n$ nodes with keys $Y_i = y_i G$,
we recursively define 
\[ \begin{aligned}
\textrm{shared secret}\quad
 s_i &= x_i Y_i = y_i \alpha_i \mathcomma \\
\textrm{blinding factor}\quad
 b_i &= H(\alpha_i,s_i) \mathcomma \\
\textrm{next private key}\quad
 x_{i+1} &= b_i x_i \mathcomma \\ % \quad\textrm{and} \\
\textrm{next public key}\quad
 \alpha_{i+1} &= b_i \alpha_i = b_i x_i \quad\textrm{for $i < n$.} \\
\end{aligned} \]
Our $i$th node replaces $\alpha_i$ by $\alpha_{i+1}$.

\smallskip

We therefore ask if any post-quantum public key exchanges admit 
suitable key blinding tricks similar to Sphinx. 
The answer appears to be {\bf no}, for much similar reasons to 
why these primitives fail to yield convenient signature schemes. 

There are blinding operations but they incur significant costs 
that are asymptotic in the number of hops.
%
In SIDH, a public key should not posses enough information to compute
another isogeny whose kernel consumes torsion of the same prime. 
As a result, attempts to blind SIDH keys for signature schemes add
yet another torsion prime, increasing the size of the base field.
%
In Ring-LWE, there is enough flexibility for blinding constructions,
again increasing the key size, and indeed a primitive similar to
universal reencryption exists \cite{963628}.

We declare such schemes unsuitable for another reason though: 
%
If all blinding keys $b_i$ are equally likely, then the security of
elliptic curve key blinding depends only upon the security of $s_i$.
%
Any similar statement for Ring-LWE or SIDH appear to require invoking
their underlying security assumptions in a second and perhaps different way.

As a result, any hybrid variant of Sphinx still depends upon its
post-quantum primitives' underlying security assumptions! 
%
We consider this a dramatic sacrifice since both SIDH and Ring-LWE
remain quite young, suggesting that either could be broken well
 before the invention of quantum computers.

\section{Solution}

We could adopt a more naive mix network packet format that unwraps
a new public key with each hop.  In this vein, a circuit based approach
like Tor at least avoids transmitting unnecessary key material, but
risks exposing circuit metadata in the process. 

Instead, we draw inspiration from the Axolotl ratchet % \cite{Axolotl}
which remains secure against Shor's algorithm if
 first instantiated with a single post-quantum key exchange. 

Axolotl itself continues using the same public key until witnessing
 the other side reply with it, which meshes poorly with the mix networks
for several reasons.
%
First, we prefer to exploit the the elliptic curve point
 already in our Sphinx header, which must change with each packet.
%
Second, we cannot oblige mix nodes to directly communicate with
 the sender, as that becomes onion routing and Tor's teretory.
%
Instead, we hope to improve anonymity over Tor alone by
 judiciously exploiting high latency and not creating circuits per se.

We resolve these tensions by ``swapping the order'' of the hash iteration
ratchet and the two-step ECDH ratchet that make up Axolotl.  

In fact, we shall exploit the ECDH operation already in Sphinx rather
 than adding a true two-step ECDH ratchet.
Although the mix node's key is not ephemeral, any Sphinx mix network
should rotate node keys anyways for constant-time replay protection.
A typical rotation period should be a week or month, sometimes
 faster than the reply rate seen in Pond or Signal.

\smallskip

\noindent {\bf Xolotl ratchet description :} 

\def\cn{\textrm{cn}}
\def\ck{\textrm{ck}}
\def\DH{\textrm{DH}}
\def\lk{\textrm{lk}}
\def\mk{\textrm{mk}}
\def\sk{\textrm{sk}}
\def\ECDH{\textrm{ECDH}}

Initially, a node should decode a Sphinx packet as usual, 
by producing the shared secret $s_i$, and 
unwrapping one layer of the header's onion,
 to check for a ratchet flag. 
If not found, then we continue with Sphinx, 
 extracting the next hop, MAC, and unwrapping a payload layer.
If found, then we extract the ratchet instructions instead.

These ratchet instructions consist of
 a chain name $\cn$ and chain index $j$,
along with the length the length of the previous chain, or
 perhaps other information for closing the previous chain.  

We define
\[ \begin{aligned}
\textrm{chain start}\quad
 \ck_0 &= H(\textrm{``Start''} \,||\, \sk) \mathcomma \\
\textrm{chain name}\quad
 \cn &= H(\textrm{``Name''} \,||\, \ck_0) \mathcomma \\
\textrm{chain keys}\quad
 \ck_{j+1} &= H(\textrm{``Chain''} \,||\, \ck_j) \mathcomma \\
\textrm{link keys}\quad
 lk_j &= H(\textrm{``Link''} \,||\, \ck_j) \mathcomma \\
\textrm{packet keys}\quad 
 s' &= H(\lk_j \,||\, s_i) \mathcomma \\ % \quad\textrm{and} \\
\textrm{source keys}\quad 
 \sk_j &= H(\textrm{``Source''} \,||\, s') \mathperiod \\
\end{aligned} \]
In essence, there is a hash iteration ratchet named by $\cn$
from which we determine a link key $\lk$.
% the internal state $(\ck, (a,\lk_a), \ldots, (b,\lk_b))$
We hash $\lk$ with the Sphinx shared secret $s_i$ to produce
an altered Sphinx shared secret $s'$.
We finally conclude by running Sphinx with $s'$ to produce the 
blinding factor $b_i$ and unwrap both the Sphinx header and body. 

\begin{figure}[th!]
\begin{tikzcd}[ampersand replacement=\&, column sep=small]
\cdot \ar[r] \& \cdot \ar[r] \ar[d] \& \cdot \ar[r] \ar[d] \& \cdot \ar[r] \ar[d] \& \cdot \ar[r] \ar[d] \& \ck \ar[r, dotted] \& ? \& \\
 \& \lk \ar[d] \& \lk \ar[d] \& \lk \ar[d, dotted] \& \lk \ar[d] \&  \& \& \\ 
 \& \ECDH \ar[d] \& \ECDH \ar[d] \& ? \& \ECDH\ar[d] \&  \& \& \\
 \& \mk \& \mk \ar[dddll, in=90, out=270] \&  \& \mk \ar[dddllll, dotted, in=30, out=270] \&  \& \& \\
\\
\\
\cdot \ar[r] \& \cdot \ar[r] \ar[d] \& \cdot \ar[r] \ar[d] \& \cdot \ar[r] \ar[d] \& \cdot \ar[r] \ar[d] \& \ck \& \& \\
 \& \lk \ar[d] \& \lk \ar[d] \& \lk \ar[d] \& \lk \ar[d] \&  \& \& \\ 
 \& \ECDH \ar[d] \& \ECDH \ar[d] \& \ECDH \ar[d] \& \ECDH\ar[d] \&  \& \& \\
 \& \mk \ar[dddl, in=90, out=270] \& \mk \ar[dddll, dotted, in=50, out=270] \& \mk \ar[dddlll, dotted, in=30, out=270] \& \mk \&  \& \& \\
\\
\\
\cdot \ar[r] \& \cdot \ar[r] \ar[d] \& \cdot \ar[r] \ar[d] \& \cdot \ar[r] \ar[d] \& \cdot \ar[r] \ar[d] \& \ck \& \& \\
 \& \lk \ar[d] \& \lk \ar[d] \& \lk \ar[d] \& \lk \ar[d] \&  \& \& \\ 
 \& \ECDH \ar[d] \& \ECDH \ar[d] \& \ECDH \ar[d] \& \ECDH\ar[d] \&  \& \& \\
 \& \mk \& \mk \& \mk \& \mk \&  \& \& \\
\end{tikzcd}
\end{figure}

\smallskip

Assuming an initial ratchet source was created using a post-quantum
key exchange, an Xolotl ratchet retains that post-quantum security.
It also provides an interesting measure of forward secrecy against
a classical adversary. 

For these advanage, we have exposed to the node $i$ that all packets
using this particular ratchet have either the same sender or receiver.

There is no need for every hop to employ a ratchet though, but
 specific usage patterns require detailed analysis. 

\[ \begin{aligned}
\textrm{User} \to &\textrm{Tor} \to \textrm{Xolotl} \to \textrm{Sphinx} \to \\
\quad &\to \textrm{Xolotl} \to \textrm{Sphinx} \to \textrm{Cross} \to \cdots 
\end{aligned} \]

In particular, we believe that retaining ratchets for an extended period
could offer greater forward secrecy than reenstanciating them all with
each session.  There are many concerns around this point however, like 
ratchet ...

Aside from usage pattern and ratchet longevity,
there several interesting possible enhancements worth discussing : 
\begin{itemize}
\item
An onion header could contain a SIDH key accessed by a single mix node,
allowing for post-quantum security at one middle hop
 where no ratchet exists.
\item

\item 
A standard hash iteration ratchet could occupy considerable space
 if early message see long delays. 
We believe a tree-like hash ratchet configuration would be more efficient,
 especially if ratchets are used with SURBs.
\item

\end{itemize}



% \section*{Acknowledgements}
% This work benefits from the financial support of the Brittany Region
% (ARED 9178) and a grant from the Renewable Freedom Foundation.


%\newpage

\bibliographystyle{abbrv}
\bibliography{mix,pq,rlwe,sidh}

\end{document}


\section{}


