

\documentclass[twoside,letterpaper]{sig-alternate}
\usepackage[margin=1in]{geometry}
\usepackage[utf8]{inputenc}
\usepackage{url}
\usepackage{eurosym}
\usepackage{tikz}
%\usepackage{listings}
%\usepackage{graphicx}
%\usepackage{wrapfig}
%\usepackage{caption}
%\usepackage{subcaption}

\usetikzlibrary{cd}
%\usetikzlibrary{shapes,arrows}
%\usetikzlibrary{positioning}
%\usetikzlibrary{calc}

\def\mathcomma{,}
\def\mathperiod{,}

\title{Xolotl ratchet}
\subtitle{A selectively stateful mixnet format for post-quantum security}
\author{Jeffrey Burdges}
\date{\today}

\begin{document}
\maketitle

% \section{}

% L\'aszl\'o Baba's quasipolynomial time algorithm for graph isomorphism\cite{Babai-GI}

We describe a new double ratchet construction Xolotl,
 inspired by the Axolotl ratchet, % \cite{Axolotl},
that integrates with the Sphinx mix network packet format \cite{Sphinx}.
We argue this opens the door to compact mix network formats with
truly hybrid anonymity, meaning they rests upon the stronger of the
security assumptions required by the different public key primitives employed.


\section{Problem} % Post-quantum key sizes

Amonst the chief obstacles to deploying postquantum cryptography are
the comparatively large key sizes.  As a comparison : 
%
A recent Ring-LWE key exchange New Hope \cite[\S7, p.10]{NewHope} needs
 key sizes of 1824 or 2048 bytes, both of which must be ephemeral,
while a current McElise-like system McBits % \cite{McBits,InitRec}
 needs a stagering 1 megabyte public keys.
%
Supersingular isogeny Diffie-Hellman (SIDH) \cite[p. 21]{SIDH-2016} keys
are only 564 bytes, or 751 bytes uncompressed, but
 the key exchange requires at least 100 times as much CPU time as
 an ECDH key excahnge with equivelent classical security.

Anonymity tools like mix networks are sensitive to key size because 
users interact with numerous nodes and key material overhead is 
quadratic in the number of hops. % $n(n+1)/2$

\smallskip
% \section{Sphinx key blinding}

Sphinx \cite{Sphinx} is a packet format for anonymizing mix networks
that is provably secure in the universal composabilty framework, and
 addresses the key material burden by mutating or reblinding a
 single ephemeral public key $\alpha$ with each hop,
 as opposed to unwrapping an unrelated public key for each hope.

An ECC key is blinded by multiplication with a shared secret scalar
derived from the Diffie-Hellman exchange:
After selecting an intial private scalar $x_0$,
 public curve point $\alpha_0 = x_0 G$, and 
 a sequence of $n$ nodes with keys $Y_i = y_i G \textrm{shared}$,
we recusively define 
\[ \begin{aligned}
\textrm{shared secret}\quad
 s_i &= x_i Y_i = y_i \alpha_i \mathcomma \\
\textrm{blinding factor}\quad
 b_i &= H(\alpha_i,s_i) \mathcomma \\
\textrm{next private key}\quad
 x_{i+1} &= b_i x_i \quad\textrm{and} \\
\textrm{next public key}\quad
 \alpha_{i+1} &= b_i \alpha_i = b_i x_i \quad\textrm{for $i < n$.} \\
\end{aligned} \]
Our $i$th node replaces $\alpha_i$ by $\alpha_{i+1}$.

\smallskip

We therefore ask if any post-quantum public key exchanges admit 
suitable key blinding tricks similar to Sphinx. 
The answer appears to be {\bf no}, for much same reasons as 
these primitives fail to yield convenient signature schemes. 

There are blinding operations but they  incur significant costs 
that are asymptotic in the number of hops.
%
In SIDH, a public key should not posses enough information to compute
another isogeny whose kernel consumes torsion of the same prime. 
As a result, attempts to blind SIDH keys for signature schemes add
yet another torsion prime, increasing the size of the base field.
%
In Ring-LWE, there is enough flexibility for blinding constructions,
again increasing the key size, and indeed a primitive similar to
universal reencryption exists \cite{963628}.

We declare such schemes unsuitable for another reason though: 
%
If all blinding keys $b_i$ are equally likely, then the security of
eliptic curve key blinding depends only upon the security of $s_i$.
%
Any similar statment for Ring-LWE or SIDH apepar to require invoking
their underlying security assumptions a second and perhaps different way. 

As a result, any hybrid variant of Sphinx still depends upon its
post-quantum primitives' underlying security assumptions! 
%
We consider this a dramatic sacrifice since both SIDH and Ring-LWE
remain quite young, suggesting that either could be broken well
 before the invention of quantum computers.

\section{Solution}

We could adopt a more naieve mix network packet format that unwraps
a new public key with each hop.  In this vein, a circut based approach
like Tor at least avoids transmitting unecessary key material, but
risks exposing circut metadata in the process. 

Instead, we draw inspiration from the Axolotl ratchet % \cite{Axolotl}
which remains secure against Shor's algorithm if first being instanciated
with a single post-quantum key exchange. 
%
Axolotl itself cannot however expoit the elliptic curve point in
 our Sphinx header, nor be used for contacting a mix network node, 
because Axolotl continues using the same public key until witnessing
 the other side reply with it.

In fact, we should hope to improve anonymity by judiciously exploiting
higher latency, meaning the mix nodes' keys should live for some
reasonable period of time, like a week or a month.

We resolve these tensions by ``swapping the order'' of the hash interation
ratchet and the two-step ECDH ratchet that make up Axolotl.

\smallskip


\def\ck{\textrm{ck}}
\def\DH{\textrm{DH}}
\def\lk{\textrm{lk}}
\def\mk{\textrm{mk}}
\def\ECDH{\textrm{ECDH}}
\def\name{\textrm{name}}

% \begin{figure}[h!]
\[\begin{tikzcd}[ampersand replacement=\&, column sep=small]
\cdot \ar[r] \& \cdot \ar[r] \ar[d] \& \cdot \ar[r] \ar[d] \& \cdot \ar[r] \ar[d] \& \cdot \ar[r] \ar[d] \& \ck \ar[r, dotted] \& ? \& \\
 \& \lk \ar[d] \& \lk \ar[d] \& \lk \ar[d, dotted] \& \lk \ar[d] \&  \& \& \\ 
 \& \ECDH \ar[d] \& \ECDH \ar[d] \& ? \& \ECDH\ar[d] \&  \& \& \\
 \& \mk \& \mk \ar[dddll, in=90, out=270] \&  \& \mk \ar[dddllll, dotted, in=30, out=270] \&  \& \& \\
\\
\\
\cdot \ar[r] \& \cdot \ar[r] \ar[d] \& \cdot \ar[r] \ar[d] \& \cdot \ar[r] \ar[d] \& \cdot \ar[r] \ar[d] \& \ck \& \& \\
 \& \lk \ar[d] \& \lk \ar[d] \& \lk \ar[d] \& \lk \ar[d] \&  \& \& \\ 
 \& \ECDH \ar[d] \& \ECDH \ar[d] \& \ECDH \ar[d] \& \ECDH\ar[d] \&  \& \& \\
 \& \mk \ar[dddl, in=90, out=270] \& \mk \ar[dddll, dotted, in=50, out=270] \& \mk \ar[dddlll, dotted, in=30, out=270] \& \mk \&  \& \& \\
\\
\\
\cdot \ar[r] \& \cdot \ar[r] \ar[d] \& \cdot \ar[r] \ar[d] \& \cdot \ar[r] \ar[d] \& \cdot \ar[r] \ar[d] \& \ck \& \& \\
 \& \lk \ar[d] \& \lk \ar[d] \& \lk \ar[d] \& \lk \ar[d] \&  \& \& \\ 
 \& \ECDH \ar[d] \& \ECDH \ar[d] \& \ECDH \ar[d] \& \ECDH\ar[d] \&  \& \& \\
 \& \mk \& \mk \& \mk \& \mk \&  \& \& \\
\end{tikzcd}\]
% \end{figure}

Initially, we decode a Sphinx packet as usual, but unwrapping one
layer of the header's onion optionally reveals ratchet instructions. 

These ratchet instructions consist of a chain name, chain index, and
a cap on the length of the previous chain. 

 chain index $j < 256$,
 chain keys $\ck_{j+1} = H(\ck_j)$,
 link keys $lk_j = H(\ck_j)$
 message keys $\mk_j = H(\lk_j,s_i)$ 


 chain source $H(\mk_j)$
 chain name $H^2(\mk_j)$



% \section*{Acknowledgements}
% This work benefits from the financial support of the Brittany Region
% (ARED 9178) and a grant from the Renewable Freedom Foundation.


%\newpage

\bibliographystyle{abbrv}
\bibliography{mix,pq,rlwe,sidh}

\end{document}


\section{}


