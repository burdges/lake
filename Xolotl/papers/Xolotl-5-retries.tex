% Xolotl-5-retries.tex

\section{Reliability}

We must ensure reliable message delivery even when some mix nodes
become unavailable.  We should assess nodes' reliability before
publishing their new routing keys, so that more reliable nodes can
be placed into more critical network strata, especially the
aggregation points.

There are however techniques that help address nodes, or even
aggregation points becoming unavailable, like senders sending
duplicate messages, or aggregation points mirroring one another.
We view mirroring as an interesting approach, but feel it goes far
beyond the scope of our current project, as mirroring well might
entail reencryption.  Also, duplicate messages help reliability
throughout the network.


\subsection{Acknowledgements}

We cannot ensure reliable message delivery with network level
measures that operate only between adjacent hops.  Indeed, any node
may disappear from our network at any time, including with inbound
packets, and our PKI cannot even reflect the current state of the
network.

In consequence, we require acknowledgements (ACKs) that operate above
the mix network layer.  We therefore envision integrate ACKs with the
Axolotl ratchet from \S\ref{subsec:crossover} that lies above the mix
network and below applications, although one might push this higher
up into yet another layer that erasure codes data into the messages,
or even into the applications themselves. We therefore leave such 
final reliability measures as an unspecified problem for further
research.

Instead, we discuss here only incomplete reliably measures that
improve operation of the mix networks.  In particular, we observe
that relying only upon application layer acknowledgements could
dramatically increase our latency.  We should ``spend'' latency on
anonymity instead of reliability whenever possible.

\begin{issue}
How do different reliability measures impact latency?
\end{issue}

As a first step, adjacent hops should utilize a reliable transport,
meaning a hop should continue attempting to deliver a message until
it confirms the next hop received the message.  

There are several tricky issues here, like if hops report messages
dropped due to replay protection, so the exact form this takes
remains an open question \cite{??cite??all??the??things??}

...


\subsection{Cross over ACKs}

Above network level ACKs, we have an interesting mechanism for ACKs
to traverse a cross over point.  

...



... Cyclic ACKs for George ..


\subsection{Duplicate messages}

...
%TODO: Anonymity costs of duplicate messages?
%TODO: Multiple latencies


\subsection{Garlic retries}\label{subsec:garlic_retries}

We do not necessarily have much bandwidth available on the channel
between the client and the mix network.  As an example, Pond clients
support only about 288 messages per day because they wait on average
5 minutes between messages.

We can reduce the need for duplicate messages on this link if
we tell our cross over points to send retries instead, but this
requires providing it with multiple SURBs so that retries do no
violate replay protection.

In \S\ref{subsec:crossover}, we considered storing SURBs inside
$\delta$, instead of $\beta$, because $\delta$ has no remaining layers
of onion encryption from Sphinx after a cross over point decrypts it. 
As $\delta$ is large, we may encode several SURBs into $\delta$,
so as to send the same message to multiple recipients, or to send
the same message to the same recipient multiple times.  

In fact, we can achieve this more easily than sending to multiple
recipients because the recipient can supply several SURBs with the
same cross over point.

These SURBs can have different delays of course, but we can also
provide retries:  Instead of delivery to an aggregation point being
the end of our messages' path, we use a {\tt drop off} command that
delivers the packet to the aggregation point as usual, but also sends
the packet on to yet another hop, after possibly replacing the body.
The SURB executing command routes the packet back to the cross over
point, where it executes a {\tt delete} command that deletes any
pending retries.

All this works fine if we use a contact point that holds the SURBs
itself, as in \S\ref{subsec:contact_points}, instead of using a
vanilla cross over point.  In any case, recipients should supply the
SURBs in groups for {\tt delete} commands to work.

We note this scheme increases our vulnerability to attacks by the
cross over point:  At best, this {\tt delete} command reveals
dangerous information about what nodes were reachable and that our
message was being sent with redundancy, as opposed to multiple 
recipients.  Also, it assumes the reliability of both the cross over
point and any nodes on the path to it.
%
We shall discuss situations where this may be acceptable in the next section.


\subsection{Email integration}\label{subsec:LEAP}

We have focused on designing an asynchronous messaging architecture
with extremely strong privacy properties, including anonymity for both
sender and recipient, even from one another, and anti-discovery
measures for recipients' aggregation points.  
%
We consider this distributed architecture preferable overall, but
we recognize that these protections verge on excessive for many users,
that email is tricky to replace, and that email providers may have
benevolent political reasons to remain involved.

We may adopt the {\tt drop off} and {\tt delete} commands described
in \S\ref{subsec:garlic_retries} to integrate our system with email
providers, while giving senders and recipients anonymity from
providers, but not from one another.

We now assume the sender knows both the recipient's aggregation point
and their mailbox name at their aggregation point.  As a result, our
sender may build SURBs to reach the recipient without the recipient
supplying them to either the sender or a contact point.  
In this case, the sender picks the cross over point themselves, 
likely their own Email provider.

In this case, our recipient cannot already know the layers of
encryption applied to $\delta$ after the cross over point.  
We must therefore encode the cross over point's routing key, a seed,
and an epoch identifier for the list of all mix nodes.  From this,
the recipient can rederive the SURB by first rederiving the route
and then computing the usual key exchanges.  A sender encrypts this
information to the final recipient and includes that encrypted
message in the {\tt drop off} command.  A recipient's aggregation
point places this encrypted message into the SURB log when executing
{\tt drop off} command.

There is now a risk the recipient's mailbox does not exist. 
In this case, our {\tt drop off} should specify an alteration to
either $\delta$ or the SURB log, and the sender should warn their
cross over point that this alteration means the message was undeliverable.

A priori, there is considerable risk of unwanted messages like SPAM
under this scheme.  As a result, our cross over point $X$, aka the
provider, must require authentication by the sender.  
Recipients must learn the provider when they reconstruct the SURB.  
If they do not, then message decryption necessarily fails.



