% Xolotl-2-sphinx.tex

\section{Sphinx}\label{sec:sphinx}

We start with a setup consisting of a Sphinx-based mix network in
which each mix node has a medium term routing key $X = x G$ with a
predefined validity period, and a medium term name that identifies
both it and $X$.  We expect this name would be derived from the
signature of a longer term signing key on $X$ and its validity period.

For simplicity, we assume all nodes in the network are aware of the
medium term routing keys $X$ and validity periods for all other nodes.
Furthermore, public keys for future periods are expected to be
securely distributed ahead of time, possibly months before they are
used.


\subsection{Routing Sphinx packets}

We provide a rough outline of the Sphinx mix net packet format from 
\cite{Sphinx}, adapted to transport SURBs for use at a cross over point
to provide simultaneous anonymity for both senders and receivers.
For further background, we refer the reader to the Sphinx construction
in \cite{Sphinx} and the securityproof that inspired it~\cite{FormalOnion},
as well as wide block ciphers~\cite{Lionness}.

\begin{figure}
  \begin{center}
  \input{sphinx2.pdf_tex}
  \end{center}
  \caption{The Sphinx packet format (reproduced from~\cite{Sphinx}).}
  \label{fig:sphinx}
\end{figure}

A Sphinx packet (Figure~\ref{fig:sphinx}) flows along a client-selected
route $X_1,\ldots,X_n$ where $n$ is less than some small network defined
constant.  A Sphinx packet consists of an elliptic curve point $\alpha$,
a routing header $\beta$ of a fixed length $L$, a single-use reply block
a MAC $\gamma$ that authenticates the contents of $\beta$, and a body
$\delta$.  Both $\beta$ and $\delta$ are onion encrypted, but $\beta$
uses a stream cipher, while $\delta$ uses a wide block cipher since
it is not authenticated.

Each hop $X$ processes $(\alpha,\beta,\gamma,\delta)$ as follows.  
First, it computes a shared secret $s = x \alpha$ to
derive a replay code, a MAC key, a stream cipher key, 
 a blinding key $b$, and a wide block cipher key. 
It uses the MAC key to check the MAC of $\beta$ and
 then checks its database for the replay code.
It aborts and drops the packet if either the MAC check fails or
 if the replay code is present.  Otherwise it adds the replay code
 to its database.
Next, it pads $\beta$ with $l$ zeros and decrypts the result
 using the stream cipher to produce a $\hat\beta$.
An initial segment of $\hat\beta$ of length $l' < l$ must contain
a valid routing command.  

We discuss some new routing commands in this paper, but the typical
routing command states the next hop $N'$ to which to forward the
packet, and the $\gamma'$ the next hop $N'$ requires. 
In this case, our hop $N$ computes $\alpha' := b \alpha$,
extracts $\beta' := \hat\beta[l'..L+l']$, and
decrypts $\delta$ using the wide block cipher to produce $\delta'$.
Now the packet $(\alpha',\beta',\gamma',\delta')$ is forwarded to $N'$,
 according to whatever mixing rules the protocol specifies.


\subsection{Cross over points}\label{subsec:crossover}

% FIXME: needs a diagram ;-).

A classical mix node command tells the mix to act as a ``cross over''
point by replacing the Sphinx header with a SURB stored in the packet.
We expect the recipient had previously supplied this SURB to the
sender, along with information like the cross over point from which
the SURB works, so this gives the recipient control over routing, 
and provides them with anonymity, even from the sender.  
The sender selects the path to the cross over point, so they can be
anonymous too, even from the recipient.
% TODO: Citation for cross over points?

\smallskip

We observe that the cross over point's $\delta'$ has been stripped of
all onion encryption layers created by the sender because the sender
must not know anything about the packet after the cross over point,
including wide block cipher keys learned by anybody but hte recipient.
We therefore forbid $\delta'$ from being the plaintext message itself
and insist that it be protected with another layer of end-to-end
encryption.  We select this end-to-end encryption layer to prevent
cross over points from performing correlation attacks on $\delta'$.

If all SURBs are supplied to the sender by the recipient, then
we could simply provide one more wide block cipher key with which
the sender pre-encrypts $\delta'$ before applying its own onion layers.
We do not assume this however because SURBs have a limited lifespan. 
If the sender does not posses any SURBs, then they could contact the
recipient through a {\em contact point}.  These contact points would
act like cross over points, except that they would supply the SURB
themselves from a stash supplied by the recipient.
  
We shall explain contact points in \S\ref{subsec:contact_points}.
For now, we observe that that $\delta'$ must pass through cross over
and contact points without any encryption from the mix network layer. 

\begin{issue}
How is $\delta$ encrypted to hide it from the cross over points?
\end{issue}

In consequence, we must encrypt $\delta'$ with another layer that
lies above the mix net layer.  We prefer an Axolotl ratceht with
header encryption here for established contacts, as the header
encryption prevents some correlation attacks by the cross over
point, but adding new contacts will require another scheme in
\S\ref{subsec:greeting_points}.  

It follows that $\delta'$ could be encrypted in a few different ways.
As an extreme case, a bare curve point might create a tagging attack
visible to the cross over point.  
There are tools like Elligator~\cite{elligator} that make it harder
for a cross over point to dinstinguish between encrpytion schemes
used for $\delta'$, and hence the role of the message, but they
come with their own costs and constraints. 

\smallskip

We now detail how a cross over points operate : 
Initially, a {\tt cross over} command provides a length $k$ along
with new $\alpha''$ and $\gamma''$ that replace our old $\alpha$ and
$\gamma$, respectively. 
We replace $\beta$ with $\beta'' := \beta'[0..k]$, or equivalently
$\hat\beta[l'..l'+k]$, followed by $L-k$ zero bytes.
\[ \beta'' :=  \beta'[0..k] \,||\, 0\cdots0 \]
In addition, $\delta$ is replaced by the decrypted $\delta'$, 
Now the mix reruns the Sphinx decoding process on the new packet
$(\alpha'',\beta'',\gamma'',\delta')$. 
After this second run there is nothing about the packet that 
identifies it --- not even to the original sender.
We zero the trailing $L-k$ bytes of $\beta''$ before decoding so that
the recipient can compute the required MACs, including $\gamma''$.

As stated, our recipient has chosen the cross over point $X$,
supplied the sender with $(X,\alpha'',\beta'[0..k],\gamma'')$,
and the sender built a route to $X$.  We imagine $\alpha$ to be
twice the size of $\gamma$ and $\gamma$ to be roughly the length of
$X$, so the {\tt cross over} command itself consumes 1.5 hops of $\beta$.
As the recipient picked $X$, this reduces the sender's maximum route
length by roughly 2.5 hops, plus any hops in $\beta''$.
We must bound the length $k$ or else the recipient could prevent
the sender from adding enough hops themselves, what we call a
long SURB attack.

\begin{issue}
Why do cross over points run two Sphinx decoding oeprations,
one for the sender and one for the reciever?
\end{issue}

We could increase the total route length by one if we added a next
hop $X''$ to the {\tt cross over} command, built the tail of $\beta''$
using a stream cipher and sent $(\alpha'',\beta'',\gamma'',\delta')$
to $X''$. We consider this is wasteful because hops cost vastly more
than a few bytes in the header.

We observed that $\delta'$ has beed stripped of all onion encryption
layers created by the sender at the cross over point, which may
leak metadata to the cross over point.
In this scenario, it impacts our mix networks' security less if
$\delta'$ is only ever seen by the cross over point, possibly
simplifying our overall design.

In consequence, if we added this second hop $X''$ to lengthen the
route then $X''$ might discover its position as the hop after a
cross over point.  We could have a more oblivious hop at the cost
of adding a few bytes to the header; hence our design above.

\begin{issue}
How should the SURB be encoded into the packet?
\end{issue}

We have encoded the SURB into a {\tt cross over} command that lies
in $\beta$.  There are two alternative approaches to encoding the
SURB in the packet though: 

First, the SURB could be placed into a special segment called
$\epsilon$, thus avoiding the long SURB attack check.
We must now however MAC $\epsilon$ like $\beta$ or else any
node in the sender's route can ``tag'' a message allowing its
identification by the cross over point $X$.  We also zero
$\epsilon$ at the cross over point like we zero the trailing
$L-k$ bytes of $\beta''$ currently.  

We descided placing the SURB into $\beta$ buys us greater flexibility
since $\epsilon$ has become functionally equivalent to more $\beta$.  
Yet, an encoding using $\epsilon$ would reduce our stream cipher
output consumption by $|\epsilon|$, or roughly a factor of two,
on all commands besides the usual transmit command.  It follows that,
if we find the flexibility unecessary, this scheme might eventially
serve as performance optimization. 

Second, we could place the SURB into a prefix of $\delta'$ because,
as noted above, $\delta'$ has been stripped of all onion encryption
layers created by the sender.  In this approach, a packet supports
a slightly larger payload if it does not use a SURB and perhaps
multiple SURBs for multiple recipients, so called garlic encryption.
In the short term, we fear that exploiting these options complicates
our API however.  Also, this scheme complicates the layering of
message and mix encryption.  Worse, there is a tagging attack with
visibility by the cross over point if one uses this scheme.  We shall
leave the door open for this approach in future though becuase suport
for multiple SURBs remains an interesting option, and the cross over
point might be semi-trusted under some conditions.


% \subsection{Sphinx packet construction}

% TODO: briefly explain packet construction ...


