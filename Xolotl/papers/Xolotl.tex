

\documentclass[twoside,letterpaper]{llncs}
% \documentclass[twoside,letterpaper]{sig-alternate}

\usepackage{amsmath}
\usepackage[margin=1in]{geometry}
\usepackage[utf8]{inputenc}
\usepackage{url}
\usepackage{eurosym}
\usepackage{tikz}
%\usepackage{listings}
%\usepackage{graphicx}
%\usepackage{wrapfig}
%\usepackage{caption}
%\usepackage{subcaption}

\usetikzlibrary{cd}
%\usetikzlibrary{shapes,arrows}
%\usetikzlibrary{positioning}
%\usetikzlibrary{calc}

\def\mathcomma{,}
\def\mathperiod{.}

\def\mathcomma{}
\def\mathperiod{}


\title{Xolotl ratchet}
\subtitle{A selectively stateful mixnet format for forward secure and post-quantum anonymity}
\author{Jeffrey Burdges}
\date{\today}

\begin{document}
\maketitle

% \section{}

% L\'aszl\'o Baba's quasi-polynomial time algorithm for graph isomorphism\cite{Babai-GI}

\begin{abstract}
We describe a new double ratchet construction Xolotl,
 inspired by the Axolotl ratchet \cite{Axolotl},
that integrates with the Sphinx mix network packet format \cite{Sphinx}.
We argue this opens the door to compact mix network formats with
stronger forward secrecy and truly hybrid anonymity, meaning they
rest upon the stronger of the security assumptions required by
 the different public key primitives employed.
\end{abstract}


% \section{Motivation}

Anonymity systems based on ``onion routing'' \cite{??OR??}
like Tor or I2P are known to be vulnerable to correlation attacks by
a passive adversary who can observe both endpoints of a circuit, such
as a national ISP.   Any attempt to defeat correlation attacks must
take latency into consideration. 

There are several recent proposals like \cite{Alpenhorn??} and
\cite{Dissent??} that avoid introducing much latency by instead
introducing vast amounts of unnecessary traffic, which impacts their
scalability, and limits their anonymity set size.  We consider such
approaches problematic because schemes that do not scale well would
make privacy expensive, and might favor powerful players seeking to
avoid accountability over efforts to liberate people from power. ???

We favor the opposite trade off in which we impose higher latency but
avoid introducing excessive cover traffic. In effect, we propose to
sacrifice use cases that require low latency like voice, while
offering an inexpensive privacy tool that defeats correlation attacks.
for high latency applications, including text messaging applications.

In doing so, there is a cryptographic inconvenience that high latency
schemes like mix networks lack forward secrecy.  We give low latency
anonymity tools like Tor forward secrecy by employing an ephemeral
key exchange on both servers and clients, but high latency schemes
require the client to encrypt to a long term keys of the servers.

There is a superficial similarity between forward secrecy and
post-quantum cryptography:  As post-quantum public key primitives
remain young, post-quantum protocols should be analyzed in a hybrid
setting where even ephemeral keys might be compromised.  
In other words, either the classical elliptic curve key exchange or
the post-quantum key exchange might be compromised, but not both. 
In this setting, there are technical obstacles to deploying a
post-quantum key exchange in a mix network.  

In this article, we propose Xolotl, a stateful "ratchet" based
solution, inspired by Trevor Perrin's Axolotl ratchet \cite{Axolotl},
that extends the Sphinx mix net packet format \cite{Sphinx}.
Xolotl provides limited post-quantum protections and forward secrecy
in exchange for a small sacrifice of some correlating information,
but far less than using a circuit based system like Tor.


\section{Problems} % explanation

\subsection{Forward security}

We have observed that high latency anonymity schemes depends upon
node using non-ephemeral key material, thus opening them to key
compromise attacks that do not

There are however several ways one might harden mix networks against node
keys being compromise. 

As a rule, mix networks must prevent replay attacks, normally using
a database of some form.  As this database must not grow indefinitely,
mix nodes must rotate their keys periodically.  These keys should be
destroyed when no longer used to provide a measure of forward security.

In principle, we could rotate keys quickly so that the compromise
window for each key stays short.  In general, mix networks must
support single-use reply blocks (SURBs) so that anonymous users can
receive messages.  These SURBs are built to a specific set of node
keys, so they cannot outlive the key rotation schedule.  This creates
a tension between forward security and usability.

We could rotate node keys slow enough for our usability goals, but add
a faster rotating identity-based epoch key issued by a collective of
trusted nodes, and deform the SURB's key material to account for this
rotating identity-based epoch key.  This buys use forward security
assuming at least some trusted node does not get compromised, but
sounds difficult to deploy in practice. 

We could use punctured encryption \cite{punctured_encryption} for our
mix nodes key, so that mix nodes who correctly puncture their key
after decrypting a message cannot decrypt the same message again. 
As a rule, punctured encryption schemes require $O(n)$ time for
decryption where $n$ is the number of punctures so far.  There are
techniques for epoch based puncturing that make $n$ far less than
the number of packets.  In a mix network, these would require
deforming the SURB as well. 

Any scheme for deforming SURBs requires a delicate proof of security
because several mix network packet formats based on malleable key
material were broken in \cite{GD-breaking_universal_reencryption}.  
Also, these ideas all require slower pairing based cryptography that
increase our mix nodes vulnerability to denial of service attacks. 

George Danezis' fs-mixes \cite{fs-mix} provide an approach that
avoids imposing constraints on the public key exchange.  In fs-mixed,
a mix node derives a $(k,v)$ from from each key exchange and records
$(k,v)$  in a name-value store.  We already do something similar for 
replay protection, but this new name-value store outlives our public
key rotation schedule.  A subsequent packet sent in another public
key epoch may ask that $v$ be hashed into the key exchange after
revealing its name $k$.  

These fs-mixes could be viewed as a form of "ratchet" not wholly
unlike off-the-record (OtR) messaging ratchet \cite{??OtR??},
sometimes described as a three-step Diffie-Hellman ratchet.  
We shall extend fs-mixes to provide stronger protection and be more
reliable.  Intuitively this parallels how Trevor Perrin's Axolotl 
ratchet extends OtR with a hash iteration ratchet. 

\subsection{Post-quantum}

% Along with the primitives being relatively young and poorly explored,
An important obstacle to deploying post-quantum cryptography is
the comparatively large key sizes.  As a comparison : 
%
A recent Ring-LWE key exchange New Hope \cite[\S7, p.10]{NewHope} needs
 key sizes of 1824 or 2048 bytes, both of which must be ephemeral,
while one McEliece-like system McBits % \cite{McBits,InitRec}
 needs a staggering 1MB for public keys.
%
Super-singular isogeny Diffie-Hellman (SIDH) \cite[p. 21]{SIDH-2016} keys
are only 564 bytes, or 751 bytes uncompressed, but
 the key exchange requires at least 100 times as much CPU time as
 an ECDH key exchange with equivalent classical security.

Anonymity tools like mix networks are sensitive to key size because 
users interact with numerous nodes and key material overhead is 
quadratic in the number of hops. % $n(n+1)/2$

\smallskip
% \subsubsection{Sphinx key blinding}

Sphinx \cite{Sphinx} is a packet format for anonymizing mix networks
that is provably secure in the universal composability framework, and
 addresses the key material burden by mutating or reblinding a
 single ephemeral public key $\alpha$ with each hop,
 as opposed to unwrapping an unrelated public key for each hop.

In Sphinx, an elliptic curve point is blinded by multiplication with
a shared secret scalar derived from the Diffie-Hellman exchange using
the same point:
After selecting an initial private scalar $x_0$,
 public curve point $\alpha_0 = x_0 G$, and 
 a sequence of $n$ nodes with keys $Y_i = y_i G$,
we recursively define 
\[ \begin{aligned}
\textrm{shared secret}\quad
 s_i &:= x_i Y_i = y_i \alpha_i \mathcomma \\
\textrm{blinding factor}\quad
 b_i &:= H(s_i) \mathcomma \\
\textrm{next private key}\quad
 x_{i+1} &:= b_i x_i \mathcomma \\ % \quad\textrm{and} \\
\textrm{next public key}\quad
 \alpha_{i+1} &:= b_i \alpha_i \quad\textrm{for $i < n$.} \\
\end{aligned} \]
Our $i$th node replaces $\alpha_i$ by $\alpha_{i+1}$.

\smallskip

We ask if any post-quantum public key exchanges admit 
a key blinding operation suitable for Sphinx. 
At present, the answer appears to be {\bf no}, for similar reasons to
why these primitives still lack convenient signature schemes. 
In particular, there are blinding operations but they incur significant
costs  that are asymptotic in the number of hops.

In SIDH, a public key is an isogeny whose kernel consits $p$-torsion
for $p=2$ or $3$.  It reveals guide points in the $5-p$-torsion but
must not reveal the image of any known $p$-torsion points.  
As a result, current attempts to blind SIDH keys for signature schemes
add another torsion prime beyond 2 or 3, increasing the size of the
base field.  In Sphinx, we could invent new shared $p$-torsion guide
points for blinding, thus avoiding  because blinding happens after the key exchange
establishes confidentiallity, but doing so requires building confidence
in this new cryptographic operation.  Worse, there are currently open
questions around key validation in SIDH that prevent using long term
keys \cite{??SIDH-valiadation??}, which Sphinx requires.

In Ring-LWE, there is enough flexibility for blinding constructions
that increase the key size, fully homomorphic encryption schemes that
avoid this increase, and even a primitive similar to universal
reencryption exists \cite{963628}.  In all cases however, we still
increase key sizes dramatically over existing key exchanges and we face
difficult key wrap problems with the fully homomorphic encryption
schemes.

\smallskip

We consder such schemes currently unsuitable for another reason though: 

In Sphinx, there is no requirement that the blinding keys $b_i$ be
drawn from a uniform distribution because standard assumptions on
elliptic curves suitable for cryptography ask that an adversary has
no appreciable advantage in determining the correspondence between
$\alpha_i$ and $\alpha_{i+1}$ without knowing $b_i$.  
Indeed, the curve25519 function commonly used for Diffie-Helamn key
exchange in Sphinx implementations goes so far as to set the high bit
of the private scalar as a protection against non-constant time
implementation.
There are also cryptographic algorithms like signatures for which
this assumption does not suffice and blinding scalars must be chosen
from a uniform distruction.

There are similar assumptions underlying both Ring-LWE and SIDH.
We believe they should be viewed as stronger than simply assuming the
security of the key exchagne scheme itself, partially due to their
youth, but also because the underlying operations cannot so easily
yield a uniform distribution on public keys.  We believe further
research into signature schemes might help build our confidence in
blinding operation with Ring-LWE and SIDH, but this will take time.

As these schemes remain young, we want post-quantum public key
systems to be used in a hybrid protocol with an elliptic curve scheme,
so that an adversary must break both.  There is an imminent danger
that blinding operations can fail if only one of the component schemes
fails.

As an example, we consider a Sphinx-like mix net packet format that
employs the ordinary curve25519 key exchange for blinding along side
a post-quantum scheme with any magical properties we desire.  
A quantum computer can break this by computing $b_{\alpha,\alpha'}$
such that $\alpha' = b_{\alpha,\alpha'} \alpha$ for every pair of
incoming $\alpha$ and outgoing $\alpha'$.  Any $b_{\alpha,\alpha'}$
with high bit zero cannot be correct, thus giving the adversary at
least a 50\% advantage in guessing packet routes correctly.

We can correctly this by using a scalar multiplication that does not
zero the high bit in the scalar.  In doing so, we must keep the
multiplication by the cofactor so that the adversary cannot tag
packets by injecting small subgroup elements that do not impact the
key exchange itself.  These are standard moves in elliptic curve 
cryptography that work since the scalar multiplication is bijective.
To be blunt, the elliptic curve scalar multiplication can act as
a blinding operation in a hybrid protocol because the blinding itself
can be made information theoretically secure.

We enconter a fatal difficult if imagine that our post-quantum key 
exchange might be broken while elliptic curves remains secure.  
There are no known blinding operations that provide information
theoretic security, so attacks on the post-quantum component can
yield an advantage in guessing packet routes.


\section{Solution}

We could build a hybrid protocol if our post-quantum key exchange
used a seperate keys for each hop instead of blinding.  These keys
could be packaged into $\beta$ along with the routing directions and
MACs for Sphinx, but doing so makes the total size of post-quantum 
key material quadratic in the number of hops, and these post-quantum
keys are already extremely large.
In this vein, a circuit based approach like Tor at least avoids
transmitting unnecessary key material, but it exposes circuit
metadata by doing so. 

Instead, we draw inspiration from Trevor Perrin's Axolotl ratchet
\cite{Axolotl}, and George Danezis' fs-mixes \cite{fs-mix}.
\footnote{In fact, we developed Xolotl without knowing about fs-mixes
but the comparison clearly simplifies the exposition.}

% fs-mixes refresher

In George Danezis' fs-mixes, every packet leaves a trace of symmetric
key materials $v$ that {\it one} later packet can reference with its
name $k$.  In principle, we could chain these traces similarly to
how off-the-record messaging \cite{OtR??} chains key exchanges.
In fact, we could buy ourselves a measure of post-quantum protections
through chaining if our initial trace pairs $(k,v)$ were created with
a post-quantum key exchange.  Yet, we cannot employ chaining with
fs-mixes directly because doing so magnifies any unreliability in
our mix network. 

% Axolotl refresher

\def\ck{\texttt{ck}}
\def\rk{\texttt{rk}}
\def\mk{\texttt{mk}}

The Axolotl ratchet \cite{Axolotl??} replaces the off-the-record
messaging ratchet's three-step process with a two-step procedure
that omits advertising a new key update.  Instead, Axolotl continues
using the same public key $A = a G$ in each packet until witnessing
the other side change their key $B = b G$.  Axolotl does the key
exchange between the sender's new public key $A$ and receiver's
public key $B$ that was last seen by the sender.

As in OtR, there is a {\it root key} $\rk$ that Axolotl advances by
hashing it with the results of this key exchange.  Axolotl avoids using the same key for
repeated encryptions when the receiver remains silent by deriving the
{\it chain key} $\ck_i$ for a hash iteration ratchet from the root key.

\[ \begin{aligned}
\textrm{root advance}\quad
r' &:= H(r \,||\, a B) = H(r \,||\, b A) \\ % \mathcomma
\textrm{chain start}\quad
 \ck_0 &:= H(\textrm{``Start''} \,||\, \rk) \\
\textrm{chain advance}\quad
 \ck_{i+1} &:= H(\textrm{``Chain''} \,||\, \ck_i)  \\
\textrm{message keys}\quad
 \mk_{j+1} &:= H(\textrm{``Message''} \,||\, \ck_i)  \\
\end{aligned} \]

These ratchet constructions like Axolotl and OtR remain secure against
Shor's algorithm {\it if} the root key is first instantiated with some
post-quantum key exchange, and is twice the desired security level. 
In this case, our ratchet does not provide any post-quantum forward
security per se, but that might be perfectly acceptable given current
obstacles to building quantum computers, and the large key sizes of
some post-quantum key exchanges. 


Axolotl meshes poorly with the mix networks for several reasons.
%
First, we would prefer to exploit the elliptic curve point
 already in our Sphinx header, which must change with each packet,
while Axolotl must hold this constant until recieving a response.
%
Second, we cannot make mix nodes to respond to the sender, as that
becomes an onion routing like scheme and sounds more vulnerable to
traffic analysis.

\def\cn{\texttt{cn}}
\def\DH{\texttt{DH}}
\def\lk{\texttt{lk}}
\def\sk{\texttt{sk}}
\def\ECDH{\textrm{ECDH}}

We resolve these tensions by ``swapping the order'' of the hash
iteration ratchet and the two-step Diffie-Hellman ratchet that make
up Axolotl.   
%
We have not root key per se in Xolotl.  
Instead, we have only a hash iteration ratchet, called a branch,
consisting of a {\it chain key} $\ck_j$ representing the head of the
ratchet and {\it link keys} $\lk_j$, which Axolotl called a message
key.  We hash this link key $\lk_j$ with the shared secret $s_i$
derived from the key exchange in Sphinx to produce both an improved
shared secret $t_j$ for another layer of Sphinx and a {\it berry key} $r_j$. 
These berry keys act like the root key in Axolotl because if the
client determines the message passed through the network successfully
then it could spawn a new hash iteration ratchet.

We address a branch with a mid-evil naming convention: 
A branch's own proper name has the form $(f,b)$ where
 $f$ is its family name determined by its parent branch, and
 $b$ is the index of the berry from which it grew.
In essence, we address the branch as ``5th child of George'',
or $(f,b)$, because everybody already knows the great deeds
of the branch's parent ``George'', or $f$ or short. 
There is however a family name $f'$ defined by $(f,b)$ so that
once $(f,b)$ has performed their own great deeds then their
children may be addressed as $(f',\cdot)$.

\smallskip \noindent {\bf Xolotl ratchet description :} 

A node begins as if decoding a typical Sphinx packet by
 producing the shared secret $s_i$, verifying the MAC, and 
 unwrapping one layer of the header's onion encrypted $\beta$,
producing a $\beta'$,
but then pauses to check for a ratchet flag. 
If not found, then our node continues with Sphinx as usual.
In particular, it takes $\beta_{i+1}$ to be $\beta'$,
derives the blinding factor $b_i$,
blinds the public key as $\alpha_{i+1} := b_i \alpha_i$,
extracts the next hop $n_{i+1}$ and MAC $\gamma_{i+1}$,
unwraps an onion layer from the payload $\delta_i$,
 yielding $\delta_{i+1}$, and queues the new packet
$(\alpha_{i+1},\gamma_{i+1},\beta_{i+1},\delta_{i+1})$ for $n_{i+1}$.

If found, then we extract the ratchet instructions instead.
These ratchet instructions consist of a branch address,
which contains a branch family name $f$ and a berry index $b$,
 as well as a chain index $j'$,
along with an intermediate MAC $\gamma'$.
% perhaps along with information for closing the previous chain. 

If the branch $(f,b)$ is unknown, then we locate its parent's
proper name $(f_0,b_0)$ using $f$ alone, and extract the berry
key $r$ with berry index $b$ on $(f_0,b_0)$,
 aborting if it does not exist.
From $r$, we define
\[ \begin{aligned}
\textrm{family name}\quad
 f' &:= H(\textrm{``Family''} \,||\, r) \quad\textrm{and} \\ % \mathcomma \\
\textrm{chain start}\quad
 \ck_0 &:= H(\textrm{``Start''} \,||\, r) \mathperiod \\
\end{aligned} \]
On our branch $(f,b)$, we inductively define 
\[ \begin{aligned}
\textrm{chain keys}\quad
 \ck_{j+1} &:= H(\textrm{``Chain''} \,||\, \ck_j) \quad\textrm{and} \\ % \mathcomma \\
\textrm{link keys}\quad
 \lk_j &:= H(\textrm{``Link''} \,||\, \ck_j) \mathperiod \\
\end{aligned} \]
In advancing the chain to index $j'$, we naturally save any 
intermediate $\lk_j$ for later use by out of sequence packets.

We now have the particular $\lk_{j'}$ requested by our ratchet 
instructions $(f,b,j')$, so we may define 
\[ \begin{aligned}
\textrm{packet keys}\quad 
 s' &:= H(\lk_j \,||\, s_i) \quad\textrm{and} \\ % \mathcomma \\
\textrm{berry keys}\quad 
 r' &:= H(\textrm{``Berry''} \,||\, s') \mathperiod \\
\end{aligned} \]
We first replace $s_i$ by $s'$ in our Sphinx-like component
and verify the intermediate MAC $\gamma'$.  If this verification
fails, then we abort and abandon our ratchet database transaction.
We thus assume the verification succeeds.  

We now save $r'$ as the berry key for $(f,b,j')$, and commit any
other ratchet database changes, including erasing $r$ or any 
$\ck_i$ previously recorded.  Our branch $(f,b)$ has started
earning the short name its children will use.  
Also, our Sphinx-like component continues, but now using $s'$.
To be specific, it derives the blinding factor $b_i$,
blinds the public key as $\alpha_{i+1} = b_i \alpha_i$,
unwraps a second layer of $\beta$ to extracts the
 next hop $n_{i+1}$ and MAC $\gamma_{i+1}$ and 
 producing our $\beta_{i+1}$,
unwraps an onion layer from the payload $\delta_i$,
 yielding $\delta_{i+1}$, and queues the new packet
$(\alpha_{i+1},\gamma_{i+1},\beta_{i+1},\delta_{i+1})$ for $n_{i+1}$.

\begin{figure}[b!]%[h!]
\begin{tikzcd}[ampersand replacement=\&, column sep=small]
\cdot \ar[r] \& \cdot \ar[r] \ar[d] \& \cdot \ar[r] \ar[d] \& \cdot \ar[r] \ar[d] \& \ck \ar[r, dotted] \& ? \& \\
 \& \lk \ar[d] \& \lk \ar[d]  \& \lk \ar[d] \&  \& \& \\ 
 \& \ECDH \ar[d] \& \ECDH \ar[d] \& \ECDH\ar[d] \&  \& \& \\
 \& r \& r \ar[dddll, in=90, out=270] \& r \ar[dddlll, dotted, in=30, out=270] \&  \& \& \\
\\
\\
\cdot \ar[r] \& \cdot \ar[r] \ar[d] \& \cdot \ar[r] \ar[d] \& \cdot \ar[r] \ar[d] \& \cdot \ar[r] \ar[d] \& \ck \& \& \\
 \& \lk \ar[d] \& \lk \ar[d] \& \lk \ar[d, dotted] \& \lk \ar[d] \&  \& \& \\ 
 \& \ECDH \ar[d] \& \ECDH \ar[d] \& ? \& \ECDH\ar[d] \&  \& \& \\
 \& r \& r \&  \& r \&  \& \& \\
\end{tikzcd}
\end{figure}

\smallskip

We take $r$ to be 256 bits so that, assuming the initial ratchet
source was created using a post-quantum key exchange, then our
Xolotl ratchet provides 128 bits of post-quantum security.

In addition, Xolotl provides a measure of forward secrecy against
a classical adversary in that the symmetric key information $r$,
$\ck_j$ for $j<j'$, $\lk_{j'}$, and $s'$ all gets erased when
the packet gets queued.  An adversary who later compromises the
private key of our node $n_i$ learns the next hop only for packets
that did not employ any ratchet at $n_i$.

For these advantages, we have exposed,
 to the node $n_i$ hosting the ratchet, that all packets using this
particular ratchet have either the same sender or receiver.

There is no need for every hop to employ a ratchet, % though, 
but specific usage patterns require detailed analysis. 

\[ \begin{aligned}
\textrm{User} \to &\textrm{Tor} \to \textrm{Xolotl} \to \textrm{Sphinx} \to \\
\quad &\to \textrm{Xolotl} \to \textrm{Sphinx} \to \textrm{Cross} \to \cdots 
\end{aligned} \]

An initial ``guard'' ratchet should be created anew for each session.
We believe that retaining middle ratchets for longer periods could
improve forward secrecy over creating them all anew with each session. 
There are many concerns around this point however, primarily balancing
the improved forward secrecy with the risk of linking packets across
different sessions, but
 also practical matters like ratchet storage requirements on mix nodes.  

Ratchet key material does not expire with node keys.  Indeed, even berry
keys may outlive node keys because the sender simply remembers them.
We may therefore set ratchet key expiration based upon storage concerns. 





Aside from usage patterns and ratchet longevity,
there are several interesting possible enhancements worth discussing : 

% \begin{itemize}
% \item
An onion header could contain a SIDH key accessed by a single mix node,
allowing for post-quantum security at one middle hop
 where no ratchet exists.

% \item 
A standard hash iteration ratchet could occupy considerable space
 if early messages see long delays. 
We believe a tree-like hash ratchet configuration would be more efficient,
 especially if ratchets are used with SURBs.

% \item
Ratchets could be given away to other users, thereby allowing a
collaborative form of forward secrecy, and obfuscating ownership. 
In principle, adversaries could not break middle ratchets without
breaking all cryptography up through to the messaging layer itself,
which could reasonably employ multiple post-quantum key exchanges,
 possibly even McBits.
% \end{itemize}


% \section*{Acknowledgements}
% This work benefits from the financial support of the Brittany Region
% (ARED 9178) and a grant from the Renewable Freedom Foundation.


%\newpage

\bibliographystyle{abbrv}
\bibliography{mix,pq,rlwe,sidh}

\end{document}


\section{}


