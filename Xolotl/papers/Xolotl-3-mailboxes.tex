% Xolotl-3-mailboxes.tex

\section{Mailboxes}

As mentioned in the introduction, we must never assume that users,
nor any agent they control, is reliably online.  
% ``It's an asynchronous world.''  - Moxie Marlinspike
% https://github.com/WhisperSystems/Signal-Android/blob/master/CONTRIBUTING.md
As a consequence, we must retain messages for an offline recipient
until their client requests delivery.  
At present, there are no successful secure messaging protocols
without this IMAP/POP-like {\em request-and-forward} functionality,
except arguably off-the-record messaging. 


\subsection{Aggregation points}

We expect this necessitates that messages destined for the same
recipient flow into mailboxes at a small number of {\em aggregation
points} that process delivery requests.  These aggregation points 
should learn as little as possible about recipients or senders, but
the recipient should learn the sender's identity from the protocol.
In~\cite{agl-pond-hmac}, Adam Langley explains that Pond has similar
requirements and indicates important advantages of delivery tokens
over more complex schemes, like group signatures~\cite{VLR,BBS}.

We can easily add a ``deliver to mailbox'' command to Sphinx so that
SURBs can act as a delivery token.  In fact, SURBs provide near
perfect delivery properties in sense of~\cite{warner-delivery}.
An aggregation point cannot contribute any knowledge about sender
even when colluding with the cross over point and recipient.
In addition, our SURB hides both the aggregation points and the
mailbox from the sender, which simplifies the recipient's task if
they change aggregation points and mailbox identifiers.

However, we must address a hiccup in how SURBs identify themselves
and senders.

\subsection{Unwinding SURB onion layers}

We noted that a cross over point sees the body, denoted $\delta'$
above, without any wide block cipher encryption layers.  We have
chosen our wide block cipher so that its encryption and decryption
operations provide exactly the same security guarantees and can
safely be swapped.  This conceptual swap happens at our cross over
point so that subsequent hops do not recognize themselves as being
after a cross over point.  They happily apply the decryption
operation without realizing that it further encrypts the body.
However, our recipient must somehow {\em unwind} these layers of
onion encryption, and to do so efficiently the recipient must have
a way to identify which of its SURBs was used, so that they
can select the correct set of keys.

\begin{issue}
How should recipients reconstruct the key materials needed to
unwind the encryption applied by nodes along their SURB's route?
\end{issue}

There are two basic schemes for obtaining the key material to
unwind the encryption applied to $\delta$ from mixes encrypting
during the SURB hops:

First, the recipient may {\em record} the wide block cipher keys used
under a {\it SURB name}.  Any packets at any hop has a natural SURB
name $\eta$ derived from the shared secret $s$.  As an optimization,
we could treat the arriving packet's $\alpha$ as the SURB name at
clients who expect themselves to be recipients.  Any SURB name scheme
has advantages in that it requires no additional space in the header
and provides simple authentication for arriving SURBs.  

Second, our recipient could construct their SURB using a {\em seed}
that they also encode into the ``bottom'' of the SURB itself, so
that they may {\em reconstruct} the original SURB and its keys.
There are potential advantages to encoding a seed in multi-device
scenarios where all devices of a user can decrypt messages without
further communication after sharing some initial key material.

However, to reconstruct the SURB from a seed in an evolving mix
network requires that recipients learn the packet's route, which
requires either that the route be encoded with the seed, which takes
considerable space, or else that the client can identify their view
of the mix network somehow.  If the network itself does not provide
a global consensus like Tor, then multi-device schemes become
extremely difficult.

At present, we believe both schemes to be worth supporting because
they cover fairly distinct use cases, including unusual situations
like in \S\ref{subsec:LEAP}.

In either scenario, our packet takes a predetermined route to the
recipient who unwinds the layers of encryption applied to the body
using information present on their disk and either the packet's alpha
$\alpha$, the shared secret $s$, or their $\beta'$.  

%TODO: Should this be said?
% Mixes delay messages along this route, but so far our recipient cannot change the route. 

\subsection{Redirecting messages}

We want SURBs to be used to reach the aggregation point because
this supports good delivery properties.  We also require the
aggregation point not learn anything about either the sender, or
the route used to reach the aggregation point.  We now turn our
attention to protecting the recipient from their own aggregation
points.

\begin{issue}
How should recipients fetch messages from their aggregation points?
\end{issue}

In principle, recipients could simply send the aggregation point
a packet with a flush mailbox command that launched all pending
packets along whatever remained of their route.  We foresee many
difficulties with such an approach, including simple route length
constraints and packet loss due to network churn, mix node key
lifetime, and clients changing guard nodes, any of which can cause
paths used in SURBs to become unavailable.

Instead, we propose that SURB directed messages should arrive at
an aggregation point without further routing commands.
The aggregation point then stores the incoming SURB's name $\eta$
derived from $s$ along with $\delta'$ in the user's mailbox.

We therefore need a method for our final recipient to retrieve both
$\delta'$ and the corresponding SURB name $\eta$ from her aggregation
point together. 

\subsection{SURB logs}\label{subsec:surb_logs}

We do not consider private information retrieval (PIR)
schemes~\cite{pir} because they increase complexity, invoke disparate
security properties, leak metadata through excessive bandwidth, and do
not support message deletion.
% FIXME: citations needed...

\begin{issue}
How can we send both $\delta$ and $\eta$ through the mix network?
\end{issue}

There is no way to ``squeeze'' our incoming SURB name $\eta_0$ into
the body $\delta$ because the body was previously encrypted with a
wide block cipher and thus cannot be shrunk.

Our recipient could ask the aggregation point to report the waiting
messages, and then send it separate SURBs assigned to each reported
SURB name.  We think this approach could work if the latency were
reasonably low, but the extra round trip creates a problem in a
protocol with even moderate latency. 

Instead, we propose adding a {\it SURB log} $\zeta$ to the header.
Any regular Sphinx hop encrypts $\zeta$ with more of the stream
cipher, but does not include it when verifying the MAC $\gamma$.
An aggregation point that receives fresh SURBs for a nonempty
mailbox simply places the previous SURB name $\eta_0$ into $\zeta$,
places $\delta'$ into the body, and populates the remaining fields
as usual.

For our recipient, there is an incoming SURB name $\eta_1$ that 
controls unwinding back to the state seen by the aggregation point.
We unwind $\zeta$ as well so that our recipient learns $\eta_0$ too,
and may continue unwinding back to the cross over point.
% FIXME: Pictures, please!

We do not MAC $\zeta$, so any adversary can flip bits arbitrarily
there.  As $\eta$ is uniformly random, we can easily make it large
enough to make brute forced search impossible.  An aggregation point
does however witness well-formed $\zeta$ from its incoming packets,
so an aggregation point could identify a message as coming form an
incorrect original sender.

For this reason, one should consider the mix network's sender
identification as an unauthenticated hint that simplifies proper
authentication in the body, say by supporting Axolotl header
encryption.  

We weakly authenticate the $\eta_0$ extracted from $\zeta$,
as being supplied by the aggregation point, because only the client
knows $\eta_1$, and all the layers of encryption applied to $\eta_0$.  
We imagine this weak authentication may help identify the source of
unwanted messages, or a denial of service attack, say by a malicious
contact point.


\subsection{Deletion policy}

To maximize the chance of delivery and to possibly support the user
accessing their messages from multiple devices, aggregation points
should not delete messages until they either need to reclaim the
storage space or are explicitly told to do so by the recipient.

\begin{issue}
After sending a message for final delivery, how should aggregation
points decide to resend or delete the messages?
\end{issue}

In a low latency Stop-and-Go mix, an aggregation points could be
given time-outs after which time it may retry messages which it sent
but for which it never received deletion instructions.

In the high latency scenario though, an aggregation points should
probably  avoid wasting two SURBs from the same batch on the same
message.  A message delivering a subsequent batch can then request
their deletion to avoid re-delivery.


\subsection{Unwinding guards and repeated retargeting}

In Tor, clients rotate circuits every 10 minutes, while holding
fixed their first hop, called a guard node.  This protects clients
from attackers seeking to push them onto a malicious guard
node~\cite{tor-guards}.

In a mix network, packets take much longer routes than tor cells.
There are nevertheless several reasons clients might concentrate
their traffic through a few guard nodes.  In principle, these
guard nodes might even be service providers who know the client. 
In this vein, mobile devices reduce their bandwidth usage and save
battery life by consolidating their messages through notification
servers, so mix clients running on a mobile device might do this as well.
% TODO: Any referenced on batching or other advantages?

At the same time, an aggregation point could hoard a user's messages
until the user selects a malicious guard, so users may wish to avoid
rotating guards too much.  

As a result, any recipient's guard node become a de-facto aggregation
point in that they should hold message for users who will reconnect.
Instead of a mailbox, the SURB should identify the final hop to the
client.  If a user does change the guard, then SURBs used to poll
messages from aggregation points may abandon messages waiting at the
old guard, and this may be common if the mix network has high latency.

We have noted that aggregation points need not delete messages until
told to do so.  Yet, we also envision conversing clients achieving
lower latency by sharing SURBs that arrive directly without passing
through long-term storage aggregation points.  Any such messages
could more easily be left waiting at an old guard.

\begin{issue}
What happens to messages left waiting at a guard instead of an
aggregation point?
\end{issue}

First, we can pick up messages from a guard similarly to how we pick
up message from an aggregation point by extending our SURB unwinding
process. If we require unwinding to support several retargeting
operations, then we simply allow several SURB names in $\zeta$,
 shift $\zeta$ rightward when adding a SURB name $\eta$, and
 shift $\zeta$ leftward when extracting a SURB name during unwinding.

In this scheme, we might notify our previous guard when we
eventually reconnect to the mix network, which leaks information.
We could avoid this by giving our guard some SURBs to message a
mailbox, probably when we first connect.  In this scheme, we might
store three SURB names in the final $\zeta$ for a message that was
meant to be direct.  We would store four for messages that already
passed through a mailbox.

% FIXME: Pictures, please! Especially for the iterative retargeting!
% Having a picture of the complete package format with the
% multi-\eta \zeta in a network diagram that shows an old
% guard, a new guard, a crossover and at least one (storage)
% aggregator (not in this order ;)) would be good.

Second, these concerns about notifying the guard present another
perhaps simpler solution.  We could modify the SURBs given directly
to our contacts to contain a new {\tt deliver} command which acts
like a conditional destination version of our usual {\tt transmit}
command.  A {\tt deliver} command attempts to send the message to
the user, like a {\tt transmit} command, but if it fails for too long
then it attempts to send the message to another mix node, again like
a {\tt transmit} command.  We prepare the remainder of the SURB after
the guard to eventually reach an aggregation point, where the user
may retrieve their message later.

Our second approach imposes only minor complexity on the guard node
processing, while dramatically simplifying the client's interactions
with the mix network.  Anonymity could be impacted by this choice,
albeit in minor ways.
%
In our first scheme, the guard node could possibly uses the SURBs
supplied by the user to help discover the user's aggregation point,
which sounds like a distant threat.  
%
In the our second scheme, we consume header space by using longer
SURBs throughout, or risk leaking that the SURB will attempt direct
dilevery to the cross over point.

We tentatively recommend the second simpler approach, but add support
for extended unwinding to our library regardless.


