% Xolotl-3-mailboxes.tex

\section{Mailboxes}

As mentioned in the introduction, we must never assume that users,
nor any agent they control, is reliably online.  
% ``It's an asynchronous world.''
% https://github.com/WhisperSystems/Signal-Android/blob/master/CONTRIBUTING.md
As a consequence, we must retain messages for an offline recipient
until their client requests delivery.  
At present, there are no successful secure messaging protocols
without this IMAP/POP-like {\em request-and-forward} functionality,
except arguably off-the-record messaging. 


\subsection{Aggregation points}

We expect this necessitates that messages destined for the same
recipient flow into mailboxes at a small number aggregation points
that process delivery requests.  These aggregation points should
learn as little as possible about recipients or senders, but the
recipient should learn the sender's identity from the protocol.
In~\cite{agl-pond-hmac}, Adam Langely explains that Pond has similar
requirements and indicates important advantages of delivery tokens
over more complex schemes, like group signatures~\cite{VLR,BBS}.

We can easily add a ``deliver to mailbox'' command to Sphinx so that
SURBS can act as a delivery token.  In fact, SURBs provide near
perfect delivery properties in sense of~\cite{warner-delivery}.
An aggregation point cannot learn the sender's identity, unless
colluding with the cross over point and all the mixes in the 
sender's route to the cross over point.  
In fact, even the receiver cannot learn more by colluding with
the aggregation and cross over points.  In addition, our SURB
hides both the aggregation points and the mailbox from the sender,
which simplifies the recipient's task if they change aggregation
points and mailbox identifiers.

However, we must address a hiccup in how SURBs identify themselves
and senders.

\subsection{Unwinding SURB onion layers}

We noted that a cross over point sees the body, denoted $\delta'$
above, without any wide block cipher encryption layers.  We have
chosen our wide block cipher so that its encryption and decryption
operations provide exactly the same security guarantees and can
safely be swapped.  This conceptual swap happens at our cross over
point so that subsequent hops do not recognize themselves as being
after a cross over point.  They happily apply the decryption
operation without realizing that it further encrypts the body.
However, our recipient must somehow {\em unwind} these layers of
onion encryption, and to do so efficiently the recipient must have
a way to identify which of its SURBs was used, so that they
can select the correct set of keys.

There are two basic schemes for obtaining the key material to
unwind the encryption applied to $\delta$ from mixes encrypting
during the SURB hops:

\begin{enumerate}
 \item The recipient may {\em record} the wide block cipher keys used
   under a {\it SURB name} $\eta$ derived from the shared secret $s$.
   This approach has the advantage that it requires no additional
   space in the header and provides simple authentication for arriving
   SURBs.

 \item Alternatively, our recipient could construct their SURB using a
   {\em seed} that they also encode into the ``bottom'' of the SURB
   itself, so that they may {\em reconstruct} the original SURB and
   keys.  Encoding a seed provides advantages in multi-device
   scenarios where all devices of a user can decrypt messages without
   further communication after sharing some initial key material.
   However, to reconstruct the SURB from a seed in an evolving mix
   network does require recipients to record their past knowledge of
   the mix network, which may eliminate the advantage in multi-device
   scenarios in cases where peers do not always share the same global
   view.
\end{enumerate}

In either scenario, our packet takes a predetermined route to the
recipient who unwinds $n$ encryptions using information present on
their disk and either $s$ or their $\beta'$.  Mixes delay
messages along this route, but so far our recipient cannot change
the route. 

\subsection{Redirecting messages}

We want SURB to be used to reach the aggregation point because
this gives us good delivery properties.  We also require the
aggregation point not learn anything about either the user, or
the SURB used to reach the aggregation point

In principle, recipients could simply send the aggregation point
a packet with a flush mailbox command that launched all pending
packets along whatever remained of their route.  We foresee many
difficulties with such an approach, including simple route length
constraints and packet loss due to network churn, mix node key
lifetime, and clients changing guard nodes, any of which can cause
paths used in SURBs to become unavailable.

Instead, we propose that SURB directed messages should arrive at
an aggregation point as the final destination as far as the SURB
is concerned. The aggregation point then stores  the incoming SURB's
name $\eta$ derived from $s$ and $\delta'$.
This should even be done if the aggregation point is the ultimate
recipient, for example if the user is logged out and thus the
secret keys for decryption might be unavailable.  
% Discuss: I'm not do sure about this part : 
 % This way, the user's desktop may also act as just another
 % aggregation point if the user then chooses to access his messages
 % from a mobile device.

Thus, we need a method for our final recipient to retrieve
$\delta'$ and the corresponding SURB name from the aggregation point.


\subsection{SURB logs}\label{subsec:surb_logs}

We do not consider private information retrieval (PIR)
schemes~\cite{pir} because they increase complexity, invoke disparate
security properties, leak metadata through excessive bandwidth, and do
not support message deletion.
% FIXME: citations needed...

There is no way to ``squeeze'' our incoming SURB name $\eta_0$ into
the body $\delta$ because the body was previously encrypted with a
wide block cipher and thus cannot be shrunk.

Our recipient could ask the aggregation point to report the waiting
messages, and then send it separate SURBs assigned to each reported
SURB name, but this requires an extra round trip in a protocol with
notable latency. 

Instead, we propose adding a {\it SURB log} $\zeta$ to the header.
Any regular Sphinx hop encrypts $\zeta$ with more of the stream
cipher, but does not include it when verifying the MAC $\gamma$.
An aggregation point that receives fresh SURBs for a nonempty
mailbox simply places the previous SURB name $\eta_0$ into $\zeta$,
places $\delta'$ into the body, and populates the remaining fields
as usual.

For our recipient, there is an incoming SURB name $\eta_1$ that 
controls unwinding back to the state seen by the aggregation point.
We unwind $\zeta$ as well so that our recipient learns $\eta_0$ too,
and may continue unwinding back to the cross over point.
% FIXME: Pictures, please!

We do not MAC $\zeta$, so any adversary can flip bits arbitrarily
there.  As $\eta$ is random, if it is large enough then random
collisions should be impossible, but an aggregation point sees
well-formed $\zeta$.  An aggregation point could therefore identify
a message as coming form an incorrect original sender.

For this reason, one should consider the mix network's sender
identification as an unauthenticated hint that simplifies proper
authentication in the body, say by supporting Axolotl header
encryption.  

We weakly authenticate the $\eta_0$ extracted from $\zeta$,
as being supplied by the aggregation point, because only the client
knows $\eta_1$, and all the layers of encryption applied to $\eta_0$.  
We imagine this weak authentication may help identify the source of
unwanted messages, or a denial of service attack, say by a malicious
contact point.


\subsection{Deletion policy}

To maximize the chance of delivery and to possibly support the user
accessing their messages from multiple devices, aggregation points
should not delete messages until they either need to reclaim the
storage space or are explicitly told to do so by the recipient.
However, aggregation points should be careful to avoid wasting two
SURBs from the same batch on the same message.  A message delivering a
subsequent batch can then request their deletion to avoid re-delivery.


\subsection{Unwinding guards and repeated retargeting}

In Tor, clients rotate circuits every 10 minutes, while holding
fixed their first hop, called a guard node.  This protects clients
from attackers seeking to push them onto a malicious guard
node~\cite{tor-guards}.

In a mix network, packets take much longer routes than tor cells.
There are nevertheless several reasons clients might concentrate
their traffic through a few guard nodes.  These guard nodes might
be service providers who know the client of course.  In this vein,
mobile devices reduce their bandwidth usage and save battery life
by consolidating their messages through notification servers, so mix
clients running on a mobile device might do this as well.
% TODO: Any referenced on batching or other advantages?

At the same time, an aggregation point could hoard a user's messages
until the user selects a malicious guard, so users may wish to avoid
rotating guards too much.  

As a result, any recipient's guard node become a de-facto aggregation
point, except that the SURB would contain instructions for another hop
to be taken.  If a user does change the guard, then SURBs used to poll
messages from aggregation points may leave messages left waiting at
the old guard, and this may be common due to the mix network's
high latency.

We noted that aggregation points also need not delete messages until
told to do so.  Yet, we also envision conversing clients achieving
lower latency by sharing SURBs that arrive directly without passing
through long-term storage aggregation points.  Any such messages could
more easily be left waiting at an old guard.

We can pick these messages up using the same technique of extending
unwinding with the SURB log $\zeta$ as with aggregation points.  In
fact, to allow repeated retargeting we allow multiple SURB names in
$\zeta$, shift $\zeta$ rightward when adding a SURB name $\eta$, and
shift $\zeta$ leftward when extracting a SURB name during unwinding.

In this scheme, we wind up notifying our previous guard when we
eventually reconnect to the mix network.  We could avoid this by
giving our guard SURBs to message a mailbox when we first connected.
In this scheme, we might store three SURB names in the final $\zeta$
for a message that was meant to be direct.  We would store more for
messages that already passed through a mailbox.

% FIXME: Pictures, please! Especially for the iterative retargeting!
% Having a picture of the complete package format with the
% multi-\eta \zeta in a network diagram that shows an old
% guard, a new guard, a crossover and at least one (storage)
% aggregator (not in this order ;)) would be good.


