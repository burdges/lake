% Xolotl-5-forwardsec.tex

\section{Key compromises and forward secrecy}\label{sec:forward-sec}

We have observed that high latency anonymity schemes depends upon
nodes using non-ephemeral key material, thus opening them to key
compromise attacks that do not impact onion routers like Tor.
There are several natural ways one might harden mix networks against
node keys being compromised. 


\subsection{Key rotation and replay attacks}

As a rule, mix networks must prevent replay attacks, normally using
a database of some form.  As this database must not grow indefinitely,
mix nodes must rotate their keys periodically.  These keys should be
destroyed when no longer used to provide a measure of forward security.

In principle, we could rotate keys quickly so that the compromise
window for each key stays short.  However, we also must
support single-use reply blocks (SURBs) so that anonymous users can
receive messages.  As SURBs are built to a specific set of node
keys they cannot outlive the key rotation schedule.  This creates
a tension between forward security and usability.

We could rotate node keys slow enough for our usability goals, but add
a faster rotating identity-based epoch key issued by a collective of
trusted nodes, and deform the SURB's key material to account for this
rotating identity-based epoch key.
% FIXME: citation needed (I assume somebody published this atrocity?)
This buys use forward security
assuming at least some trusted node does not get compromised, but
would be difficult to deploy in practice. 

We could use punctured encryption~\cite{libforwardsec} for our
mix nodes key, so that mix nodes who correctly puncture their key
after decrypting a message cannot decrypt the same message again. 
% TODO: Cite an older article on punctured encryption.
As a rule, punctured encryption schemes require $O(n)$ time for
decryption where $n$ is the number of punctures so far.  

There are techniques for epoch based puncturing that make $n$ less
than the number of packets.  In a mix network, these are either
equivelent to shortening node key lifetimes, or else require
deforming SURBs.  Any scheme for deforming SURBs requires a delicate
proof of security because several mix network packet formats based
on malleable key material were broken~\cite{Danezis2006}. 
Also, these ideas all require slower pairing-based cryptography that
would increase our mix node's vulnerability to denial of service attacks. 


\subsection{Post-quantum cryptography}

Along with the primitives being relatively young and poorly explored,
an important obstacle to deploying post-quantum cryptography is
the comparatively large key sizes.  As a comparison: 
%
A recent Ring-LWE key exchange New Hope~\cite[\S7, p.10]{NewHope} needs
 key sizes of 1824 or 2048 bytes, both of which must be ephemeral,
while one McEliece-like system McBits~\cite{McBits,PQ-InitRec}
 needs a staggering 1MB for public keys.
%
Super-singular isogeny Diffie-Hellman (SIDH)~\cite[p. 21]{SIDH-2016} keys
are only 564 bytes, or 751 bytes uncompressed, but
 the key exchange requires at least 100 times as much CPU time as
 an ECDH key exchange with equivalent classical security.

Anonymity tools like mix networks are sensitive to key size because 
users interact with numerous nodes and key material overhead is 
quadratic in the number of hops. % $n(n+1)/2$
% FIXME: Eh, what? How? Where? It is linear!!?!?!?!!

% FIXME: need to add a conclusion from this, i.e
% that we do not want to do PQC on every hop or
% for every message, but might be happy with
% setting up PQC-secured pairwise crypto with
% one designated hop on a path.

\subsection{Sphinx key blinding}

% FIXME: deduplicate with what we had above, most of it is duplicated
% and/or ought to have been said when we first talked about Sphinx.
% After that, this subsection can probably go away.

Sphinx~\cite{Sphinx} is a packet format for anonymizing mix networks
that is provably secure in the universal composability framework, and
 addresses the key material burden by mutating or reblinding a
 single ephemeral public key $\alpha$ with each hop,
 as opposed to unwrapping an unrelated public key for each hop.

In Sphinx, an elliptic curve point is blinded by multiplication with
a shared secret scalar derived from the Diffie-Hellman exchange using
the same point:
After selecting an initial private scalar $x_0$,
 public curve point $\alpha_0 = x_0 G$, and 
 a sequence of $n$ nodes with keys $Y_i = y_i G$,
we recursively define 
\[ \begin{aligned}
\textrm{shared secret}\quad
 s_i &:= x_i Y_i = y_i \alpha_i \mathcomma \\
\textrm{blinding factor}\quad
 b_i &:= H(s_i) \mathcomma \\
\textrm{next private key}\quad
 x_{i+1} &:= b_i x_i \mathcomma \\ % \quad\textrm{and} \\
\textrm{next public key}\quad
 \alpha_{i+1} &:= b_i \alpha_i \quad\textrm{for $i < n$.} \\
\end{aligned} \]
Our $i$th node replaces $\alpha_i$ by $\alpha_{i+1}$.


\subsection{Problems combining post-quantum cryptography and Sphinx}

We ask if any post-quantum public key exchanges admit 
a key blinding operation suitable for Sphinx. 
At present, the answer appears to be {\bf no}, for similar reasons to
why these primitives still lack convenient signature schemes. 
In particular, there are blinding operations but they incur significant
costs  that are asymptotic in the number of hops.

In SIDH, a public key is an isogeny whose kernel consists $p$-torsion
for $p=2$ or $3$.  It reveals guide points in the $5-p$-torsion but
must not reveal the image of any known $p$-torsion points.  
As a result, current attempts to blind SIDH keys for signature schemes
add another torsion prime beyond 2 or 3, increasing the size of the
base field.  In Sphinx, we could invent new shared $p$-torsion guide
points for blinding, thus avoiding
% FIXME: avoiding WHAT? correlation attacks?
because blinding happens after the key exchange
establishes confidentiality, but doing so requires building confidence
in this new cryptographic operation.  Worse, there are currently open
questions around key validation in SIDH that prevent using long term
keys~\cite{SIDH-NoValidation}, which Sphinx requires.

In Ring-LWE, there is enough flexibility for blinding constructions
that increase the key size, fully homomorphic encryption schemes that
avoid this increase, and even a primitive similar to universal
reencryption exists~\cite{963628}.  In all cases however, we still
increase key sizes dramatically over existing key exchanges and we face
difficult key wrap problems with the fully homomorphic encryption
schemes.

We consider such schemes currently unsuitable for another reason though: 

In Sphinx, there is no requirement that the blinding keys $b_i$ be
drawn from a uniform distribution because standard assumptions on
elliptic curves suitable for cryptography ask that an adversary has
no appreciable advantage in determining the correspondence between
$\alpha_i$ and $\alpha_{i+1}$ without knowing $b_i$.  
Indeed, the Curve25519 function commonly used for Diffie-Hellman key
exchange in Sphinx implementations goes so far as to set the high bit
of the private scalar as a protection against non-constant time
implementation.
% FIXME: there is a sentence missing here relating this back to
% RLWE, i.e saying that RLWE requires keys selected from uniform
% distribution. ``there are also'' here makes no sense.
% It is totally unclear what algorithms you are talking about!!!
There are also cryptographic algorithms like signatures for which
this assumption does not suffice and blinding scalars must be chosen
from a uniform distribution.

There are similar assumptions underlying both Ring-LWE and SIDH.
We believe they should be viewed as stronger than simply assuming the
security of the key exchange scheme itself, partially due to their
youth, but also because the underlying operations cannot so easily
yield a uniform distribution on public keys.  We believe further
research into signature schemes might help build our confidence in
blinding operation with Ring-LWE and SIDH, but this will take time.

As these schemes remain young, we want post-quantum public key
systems to be used in a hybrid protocol with an elliptic curve scheme,
so that an adversary must break both.  There is an imminent danger
that blinding operations can fail if only one of the component schemes
fails.

As an example, we consider a Sphinx-like mix net packet format that
employs the ordinary Curve25519 key exchange for blinding along side
a post-quantum scheme with any magical properties we desire.  
A quantum computer can break this by computing $b_{\alpha,\alpha'}$
such that $\alpha' = b_{\alpha,\alpha'} \alpha$ for every pair of
incoming $\alpha$ and outgoing $\alpha'$.  Any $b_{\alpha,\alpha'}$
with high bit zero cannot be correct, thus giving the adversary at
least a 50\% advantage in guessing packet routes correctly.

We can theoretically correct this by using a scalar multiplication that does not
zero the high bit in the scalar.  In doing so, we must keep the
multiplication by the cofactor so that the adversary cannot tag
packets by injecting small subgroup elements that do not impact the
key exchange itself.  These are standard moves in elliptic curve 
cryptography that work since the scalar multiplication is bijective.
To be blunt, the elliptic curve scalar multiplication can act as
a blinding operation in a hybrid protocol because the blinding itself
can be made information theoretically secure.

While this may suffice to secure a hybrid scheme against post-quantum
attacks against the Elliptic curve, we encounter a fatal difficulty
if we imagine that our post-quantum key 
exchange might be broken while elliptic curves remains secure.  
There are no known blinding operations that provide information
theoretic security, so attacks on the post-quantum component must be
expected to yield an advantage in guessing packet routes.


\subsection{Hybrid schemes using separate keys}

We could build a hybrid protocol if our post-quantum key exchange
used a separate keys for each hop instead of blinding.  These keys
could be packaged into $\beta$ along with the routing directions and
MACs for Sphinx, but doing so makes the total size of post-quantum 
key material quadratic in the number of hops, and these post-quantum
keys are already extremely large.
% FIXME: again, how do you get to quadratic!?!?!?!?
In this vein, a circuit based approach like Tor at least avoids
transmitting unnecessary key material, but it exposes circuit
metadata by doing so. 
Instead, we draw inspiration from the Axolotl
ratchet~\cite{TextSecure}, and fs-mixes~\cite{fs-mix}.\footnote{In
  fact, we developed Xolotl without knowing about fs-mixes but the
  comparison clearly simplifies the exposition.}

\subsubsection{fs-mixes}
% fs-mixes refresher
% FIXME: combined text that was previously spread all over
% the document here, but needs editing ...

fs-mixes~\cite{fs-mix} provide an approach that
avoids imposing constraints on the public key exchange.  In fs-mixes,
a mix node derives a $(k,v)$ from from each key exchange and records
$(k,v)$ in a name-value store.  We already do something similar for 
replay protection, but this new name-value store outlives our public
key rotation schedule.  A subsequent packet sent in another public
key epoch may ask that $v$ be hashed into the key exchange after
revealing its name $k$.  

These fs-mixes could be viewed as a form of ``ratchet'' not wholly
unlike off-the-record (OtR) messaging ratchet~\cite{OtR},
sometimes described as a three-step Diffie-Hellman ratchet.  
We shall extend fs-mixes to provide stronger protection and be more
reliable.  Intuitively this parallels how the Axolotl 
ratchet extends OtR with a hash iteration ratchet.

In fs-mixes, every packet leaves a trace of symmetric
key materials $v$ that {\it one} later packet can reference with its
name $k$.  In principle, we could chain these traces similarly to
how off-the-record messaging \cite{OtR} chains key exchanges.
In fact, we could buy ourselves a measure of post-quantum protections
through chaining if our initial trace pairs $(k,v)$ were created with
a post-quantum key exchange.  Yet, we cannot employ chaining with
fs-mixes directly because doing so magnifies any unreliability in
our mix network. 

\subsubsection{Axolotl}
% Axolotl refresher
\def\ck{\texttt{ck}}
\def\rk{\texttt{rk}}
\def\mk{\texttt{mk}}

The Axolotl ratchet~\cite{TextSecure} replaces the off-the-record
messaging ratchet's three-step process with a two-step procedure
that omits advertising a new key update.  Instead, Axolotl continues
using the same public key $A = a G$ in each packet until witnessing
the other side change their key $B = b G$.  Then, Axolotl does the DH key
exchange between the sender's new public key $A$ and the receiver's
public key $B$ that was last seen by the sender.

As in OtR, there is a {\it root key} $\rk$ that Axolotl advances by
hashing it with the results of the DH key exchange.  Axolotl avoids using the same key for
repeated encryptions when the receiver remains silent by deriving the
{\it chain key} $\ck_i$ for a hash iteration ratchet from the root key.

\[ \begin{aligned}
\textrm{root advance}\quad
r' &:= H(r \,||\, a B) = H(r \,||\, b A) \\ % \mathcomma
\textrm{chain start}\quad
 \ck_0 &:= H(\textrm{``Start''} \,||\, \rk) \\
\textrm{chain advance}\quad
 \ck_{i+1} &:= H(\textrm{``Chain''} \,||\, \ck_i)  \\
\textrm{message keys}\quad
 \mk_{j+1} &:= H(\textrm{``Message''} \,||\, \ck_i)  \\
\end{aligned} \]

These ratchet constructions like Axolotl and OtR remain secure against
Shor's algorithm {\it if} the root key is first instantiated with some
post-quantum key exchange, and is twice the desired security level. 
In this case, the ratchet does not provide any post-quantum forward
security per se, but that might be perfectly acceptable given current
obstacles to building quantum computers, and the large key sizes of
some post-quantum key exchanges. 


Axolotl meshes poorly with the mix networks for two key reasons:
\begin{enumerate}
 \item
 We would prefer to exploit the elliptic curve point
 already in the Sphinx header, which must change with each packet
 to avoid correlation attacks; 
 however, Axolotl must hold its key constant until receiving a response.
 \item
 We cannot make mix nodes to respond to the sender, as that
 be comes an onion routing like scheme and the additional traffic
 would make it much more vulnerable to traffic analysis.
\end{enumerate}

