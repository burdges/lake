% Xolotl-6-ratchet.tex

\section{The Xolotl ratchet}

\def\cn{\texttt{cn}}
\def\DH{\texttt{DH}}
\def\lk{\texttt{lk}}
\def\sk{\texttt{sk}}
\def\ECDH{\textrm{ECDH}}

We resolve these tensions by ``swapping the order'' of the hash
iteration ratchet and the two-step Diffie-Hellman ratchet that make
up Axolotl, creating what we call the Xolotl ratchet.   

We have no root key per se in the Xolotl ratchet.  
Instead, we have only a hash iteration ratchet, called a branch,
consisting of a {\it chain key} $\ck_j$ representing the head of the
ratchet and {\it link keys} $\lk_j$, which Axolotl called message
keys.  We hash link keys $\lk_j$ with the shared secret $s_i$
derived from the key exchange in Sphinx to produce both an improved
shared secret $t_j$ for another layer of Sphinx and a {\it berry key}
$r_j$.  These berry keys replace the root keys in Axolotl, with the
key difference being that once the client determines the message
passed through the network successfully, it can decide to spawn a new
hash iteration ratchet from a berry key.
Thus, the evolution of the Xolotl ratchet is driven by one party, while
the Axolotl ratchet branches based on actions driven by both parties.

% FIXME: is mid-evil supposed to be cute?
We address a branch with a mid-evil naming convention: A branch's own
proper name has the form $(f,b)$ where $f$ is its family name
determined by its parent branch, and $b$ is the index of the berry
from which it grew.  In essence, we address the branch as ``5th child
of George'', or $(f,b)$, under the assumption that everybody already
knows the great deeds of the branch's parent ``George'', or $f$ for
short.  There is, however, a family name $f'$ assigned to $(f,b)$.
Hence, once $(f,b)$ has performed its own great deed --- that is,
successfully delivered a message where the delivery was confirmed to
the sender --- then the children of $f' = (f,b)$ may be addressed as
$(f',\cdot)$.

\subsection{Birth of a Xolotl ratchet}

PQC.
...


\subsection{Formal Xolotl ratchet description} 

A node begins as if decoding a typical Sphinx packet by
 producing the shared secret $s_i$, verifying the MAC, and 
 unwrapping one layer of the header's onion encrypted $\beta$,
producing a $\beta'$, but then pauses to check for a ratchet command.
If the ratchet command is not found, then our node continues with
Sphinx as usual.  
In particular, it takes $\beta_{i+1}$ to be $\beta'$,
derives the blinding factor $b_i$,
blinds the public key as $\alpha_{i+1} := b_i \alpha_i$,
extracts the next hop $n_{i+1}$ and MAC $\gamma_{i+1}$,
unwraps an onion layer from the body $\delta_i$,
 yielding $\delta_{i+1}$, and queues the new packet
$(\alpha_{i+1},\gamma_{i+1},\beta_{i+1},\delta_{i+1})$ for $n_{i+1}$.

If a ratchet command is found, then we extract the ratchet instructions
instead.  These ratchet instructions consist of a branch address,
which contains a branch family name $f$ and a berry index $b$,
 as well as a chain index $j'$,
along with an intermediate MAC $\gamma'$.
% perhaps along with information for closing the previous chain. 

If the branch $(f,b)$ is unknown, then we locate its parent's
proper name $(f_p,b_p)$ using $f$ alone, and extract the berry
key $r$ with berry index $b$ on $(f_p,b_p)$,
 aborting if $f$ or berry $b$ on $(f_p,b_p)$ do not exist.
From $r$, we define
\[ \begin{aligned}
\textrm{family name}\quad
 f' &:= H(\textrm{``Family''} \,||\, r) \quad\textrm{and} \\ % \mathcomma \\
\textrm{chain start}\quad
 \ck_0 &:= H(\textrm{``Start''} \,||\, r) \mathperiod \\
\end{aligned} \]
On our branch $(f,b)$, we inductively define 
\[ \begin{aligned}
\textrm{chain keys}\quad
 \ck_{j+1} &:= H(\textrm{``Chain''} \,||\, \ck_j) \quad\textrm{and} \\ % \mathcomma \\
\textrm{link keys}\quad
 \lk_j &:= H(\textrm{``Link''} \,||\, \ck_j) \mathperiod \\
\end{aligned} \]
In advancing the chain to index $j'$, we save any 
intermediate $\lk_j$ for later use by out of sequence packets.

We now have the particular $\lk_{j'}$ requested by our ratchet 
instructions $(f,b,j')$, so we may define 
\[ \begin{aligned}
\textrm{packet keys}\quad 
 s' &:= H(\lk_j \,||\, s_i) \quad\textrm{and} \\ % \mathcomma \\
\textrm{berry keys}\quad 
 r' &:= H(\textrm{``Berry''} \,||\, s') \mathperiod \\
\end{aligned} \]
We first replace $s_i$ by $s'$ in our Sphinx-like component
and verify the intermediate MAC $\gamma'$.  If this verification
fails, then we abort and abandon our ratchet database transaction.
We thus assume the verification succeeds.  

We now save $r'$ as the berry key for $(f,b,j')$, and commit any
other ratchet database changes, including erasing $r$ or any 
$\ck_i$ previously recorded.  Our branch $(f,b)$ has started
earning the short name its children will use.  
Also, our Sphinx-like component continues, but now using $s'$.
To be specific, it derives the blinding factor $b_i$,
blinds the public key as $\alpha_{i+1} = b_i \alpha_i$,
unwraps a second layer of $\beta$ to extracts the
 next hop $n_{i+1}$ and MAC $\gamma_{i+1}$ and 
 producing our $\beta_{i+1}$,
unwraps an onion layer from the body $\delta_i$,
 yielding $\delta_{i+1}$, and queues the new packet
$(\alpha_{i+1},\gamma_{i+1},\beta_{i+1},\delta_{i+1})$ for $n_{i+1}$.

 \begin{figure}[b!]%[h!]
   \begin{center}
\begin{tikzcd}[ampersand replacement=\&, column sep=small]
\cdot \ar[r] \& \cdot \ar[r] \ar[d] \& \cdot \ar[r] \ar[d] \& \cdot \ar[r] \ar[d] \& \ck \ar[r, dotted] \& ? \& \\
 \& \lk \ar[d] \& \lk \ar[d]  \& \lk \ar[d] \&  \& \& \\ 
 \& \ECDH \ar[d] \& \ECDH \ar[d] \& \ECDH\ar[d] \&  \& \& \\
 \& r \& r \ar[dddll, in=90, out=270] \& r \ar[dddlll, dotted, in=30, out=270] \&  \& \& \\
\\
\\
\cdot \ar[r] \& \cdot \ar[r] \ar[d] \& \cdot \ar[r] \ar[d] \& \cdot \ar[r] \ar[d] \& \cdot \ar[r] \ar[d] \& \ck \& \& \\
 \& \lk \ar[d] \& \lk \ar[d] \& \lk \ar[d, dotted] \& \lk \ar[d] \&  \& \& \\ 
 \& \ECDH \ar[d] \& \ECDH \ar[d] \& ? \& \ECDH\ar[d] \&  \& \& \\
 \& r \& r \&  \& r \&  \& \& \\
\end{tikzcd}
% FIXME: ECDH should probably be SPHINX+? 
\end{center}
\end{figure}


\subsection{Faster chains}

We have described a simple linear hash iteration ratchet above.  
These work well in Axolotl because the sender controls the desired
order of delivery and the transport makes some effort to comply.

In Xolotl, we envision using ratchets when receiving a message 
using a single-use reply blocks (SURBs), as well as when sending.
We should expect SURBs to be used in a far more haphazard way or
indeed discarded when node keys rotate.  

Axolotl also benefits from being used between established contacts,
while mix nodes would provide Xolotl ratchets as a service, likely
for free.  We therefore worry malicious clients might ask mix nodes
to forward ratchets for extreme distances as a denial of service
attack, while the mix node cannot easily distinguish this malicious
behavior from poor SURB usage.

We address these concerns by optimizing our procedure for advancing
the chain key.  We reserve the low $l$ bits of the chain index
for a chain key that works as described above.  
If our index $j$ has a nonzero bit among its lower $l$ bits,
meaning $j \& (2^l-1) \neq 0$, then
we define
\[ \begin{aligned}
 \ck_{j+1} &:= H(\textrm{``Chain''} \,||\, \ck_j) \quad\textrm{and} \\ % \mathcomma \\
\end{aligned} \]
We reserve bit $l$ and higher for a faster tree-like {\it train key}
built using heap addressing.  If the lower $l$ bits of $j$ are
all zero, meaning $j \& (2^l-1) \neq 0$, then we define 
\[ \begin{aligned}
% \textrm{train left}\quad
 \ck_{2j} &:= H(\textrm{``Left''} \,||\, \ck_j) \mathcomma \\
% \textrm{train right}\quad
 \ck_{2j+2^l} &:= H(\textrm{``Right''} \,||\, \ck_j) \mathcomma \\
% \textrm{chain key}\quad
 \ck_{j+1} &:= H(\textrm{``Chain''} \,||\, \ck_j) \quad\textrm{and} \\
\end{aligned} \]
In both case, we could define the link key $\lk_j$ as before, but
the actual key derivations functions could differ between these two
cases, and doing so appears more efficient.


\subsection{State of a Xolotl ratchet}

What are the various maps a xolotl router needs to keep?
$f' \to (f,b)$, $(f,b) \to (j,ck_j,lk_j)$,
$(f,b,i) \to lk_i$ for $i < j$ if message not received;
left/right/train key data structure modifications.

...


\subsection{Expiration of a Xolotl ratchet}

When is which key discarded.  Be precise.

For example, I do not think we need to keep $(f,b)$ if
$f'$ has been used, but we did NOT talk about
how the message involving $f'$ tells us about
the maximum $j$ for which we need to keep
$(f,b) \to (j,ck_j,lk_j)$.

I don't even see an explicit statement that we
cannot fork two berries of the same $f'$ into ratchets...

