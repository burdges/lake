% Xolotl-1-intro

\section{Introduction}

Anonymity systems based on ``onion routing''~\cite{SS03} like Tor or
I2P are known to be vulnerable to correlation attacks by a passive
adversary who can observe both endpoints of a
circuit~\cite{timing-fc2004}, such as a national ISP.  Any attempt to
defeat correlation attacks must increase mixing through a combination
of higher latency and cover traffic, possibly including communication
with recipients besides those intended.
% TODO: George said this better.  See his Panoramix slides.
% cite https://blog.torproject.org/blog/one-cell-enough

There are several recent proposals like~\cite{Alpenhorn}
and~\cite{Dissent} that avoid introducing much latency by instead
introducing vast amounts of cover traffic.  There are several issues
with this trade off: 

% TODO: Issue 0?

First, if users can tolerate latency comfortably then they should
use a protocol that does so.  An adversary can extract metadata not
only from timing the protocol but from timing side channels, even
user behavior.  Also there is likely a synergy between anonymity from
latency and cover traffic \cite{??}.

Second, a protocol that scales poorly might favor powerful
organizations with fixed established anonymity sets, such as hiding
which executive at a company asked managers to violate some law.
Instead, we should favor users' whose anonymity needs naturally
cross organizational boundaries, like volunteer organizations,
journalists' sources, whistleblowers, protest organizers, etc.

We favor the opposite trade off in which we accept higher latency but
introduce less cover traffic, and thus choose to design a
{\em mix network} or mixnet.  % cite chaum or ?? % TODO: Add connective?

We believe this choice makes more efficent usages and should better
exploit network effects as the usage increases.  
In consequence, we sacrifice use cases that require low latency like
voice, and most modern web traffic, while offering an inexpensive
privacy tool that defeats correlation attacks.
Our target domain includes most text messaging applications and
e-mail, but tentatively exclude large file transfers. 

\subsection{Messaging API goals} 
% COMMENT: This seems more line API goles than network architecture

A central goal for our network architecture is that messaging is 
{\bf asynchronous} and must work reliably even if sender and receiver
are never online at the same time.  As a result, we ask that messages
can be stored for days to months at nodes in the network.  For this,
our receiver's client selects a set of nodes called {\em aggregation
points} to store incoming messages, and selects a replication level
to achieve the desired level of reliability.  Senders might learn the
number of replicas, but only the receiver knows what aggregation
points host them.  Aggregation points do not know of each other, and
collaborating aggregation points cannot detect that they are storing
the same message or message for the same receiver.

We require that senders have a way to securely resolve a new
recipient's address to a set of sender-specified single-use reply
blocks (SURBs), the usualy way that mix networks provide receiver
anonymity.  Approaches for name resolution include an out-of-band
exchange, using name resolution in the GNU Name System~\cite{gns}
using a private label, and other discussed below.  Once a set of
SURBs has been established between sender and receiver, our protocol
maintains the connection by maintaining a SURB supply so long as
both parties remain acitve.  We employ a fall back scheme for when
one party becomes inactive for too long, but which compromises on
delivery assurances.  % GNS would leak metadata if used for this.  

For path selection, sender and receiver require the ability to select
``random'' nodes for routing.  We assume that the network offers a
Byzantine fault-tolerant random peer sampling mechanism, for example
using trusted directory servers~\cite{tordir} or using fault-tolerant
random peer sampling protocols, such as BRAHMS~\cite{brahms}.  Our
design allows peers to in principle impose further constraints on the
path selection, for example limiting the set of guard
nodes~\cite{oneguardisenough}, biasing the selection in favor of
higher bandwidth routers~\cite{findexample} or selecting nodes for
persistent replication based on advertised storage capabilities.
While those choices do matter for privacy, we consider them orthogonal
and thus outside of the scope of this work.
% Discuss: I suspect peer sampling could be biased for an epistemic
% attack on users, so this line sounds like on-going research at
% present.  We do need to sort it out thought so I'm leaving this
% right now.

\subsection{Cryptographic challenges}\label{subsec:challenges}

We desire anonymity properties that improve on Tor in all respects,
provided our system can achieve a similarly large user base.  
In this vein, we want cryptographic properties that seems equivalently
strong as well.

Low latency anonymity tools like Tor achieve {\em forward secrecy}
by employing an ephemeral key exchange on both servers and clients.
We face a cryptographic inconvenience that high latency schemes
like mix networks must ask clients to encrypt to the long term keys
of mix nodes, meaning they lack conventional {\em forward secrecy}.  

There is a superficial similarity between forward secrecy and
post-quantum cryptography: As post-quantum public key primitives
remain young, post-quantum protocols should be analyzed in a hybrid
setting where even ephemeral keys might be compromised.  
In other words, there is a chance that either the classical elliptic
curve key exchange or the post-quantum key exchange might be
compromised, but the chance of both being broken is judged lower.  
We therefore wish to produce a scheme that combines the strengths of
both classic and post-quantum primitives.  However, we face technical
obstacles to deploying a post-quantum key exchange in a mix network,
starting with the simple engeneering issue that messages tend to be
larger and computations more expensive, but including the absence
of information theoretically secure blinding operations.

In this article, we propose Xolotl, a stateful ``ratchet'' based
solution, inspired by the Axolotl ratchet~\cite{TextSecure}, that
extends the Sphinx mix net packet format~\cite{Sphinx}.  
Xolotl provides limited post-quantum protections and forward secrecy
in exchange for leaking some limited correlating information and
increased path length.  

Although harmful, this leakage, and ratchet storage costs, provide an
important parameter for mix network architects:  In any mix network,
mix node key lifetimes correspond with SURB lifetimes, so anytime
contacts do not communicate during a key epoch they must reestablish
connection through a slightly riskier channel.  We believe Xolotl
provides the flexibility needed to extend SURB lifetimes without
making mix node keys too juicy of a target for adversaries.
% George Danezis' fs-mixes~\cite{fs-mix}, or perhaps punctured encryption.

\subsection{Mixing strategy and topology}

We are mostly agnostic to mixing strategy and mix network topology in
this article and accompanying software library.

A priori, SURBs conflict with some approaches to {\em Stop-N-Go mixes}
\cite{StopNGo} in which mixes drop packets that fall outside of client
specified time window.  We expect the network coordination required
for those timing checks to be unrealistic regardless. 

We do support both packet delays being client controlled as well as
client side delay detection via heart beat messages.  In particular,
there are no conflicts between our tweaks and the {\em Poisson mixes}
of Loopix \cite{Loopix}, a recent approach to Stop-N-Go mixes.
We adopt methods for selecting delays that enforce an exponential
distribution with client secretly influencing the rate paramater
$\lambda$, which incidentally lets us support mixing strategies not
controlled by the clients without packet format changes.

We introduce several new {\em commands} that Sphinx packets may give
to mix nodes.  These may leak additional information about the
packet's purpose to the node.  There are theoretical arguments that 
mix networks with a stratified topology provides slightly better
anonymity than those with a free route topology \cite{Diaz-??}.  
Also, any weakness due to free route might compound leaks from our
additional commands.  We tentatively suggest that a stratified
network topology with these commands restricted to specific strata
might address these concerns. % which again agrees with Loopix.
We make no assumption in our implementation that rules out a free
route topology however.  

\subsection{Outline and security}

We structure this article around a sequence of design decisions.
These are each highlighted by {\tt Issue} blocks and accompanied by
a discussion of the consequences and our conclusions.  
%
We believe this approach will help parties with different interests
in deploying mix networks to find the common ground to collaborate
on one network and combine anonymity sets.
% TODO: Anything more along the lines of Greg's joke about math
% being the only subject that really records its own history?

We discuss the Sphinx packet format from \cite{Sphinx} here in
\S\ref{sec:sphinx}, elaborating upon the basic routing commands
for cross over points and SURBs.
%
We modify Sphinx in two ways that potentially require adapting
the security arguments given in \cite{FormalOnion}: 

In \S\ref{subsec:surb_logs}, we add a {\em SURB log} field to the
header to support a recipient {\em unwinding} more than one SURB
applied to the header.  At first blush, the SURB log appears rather
frightening but it should not impact security arguments too much.

A prioi, our scheme for processing {\em commands} does not represent
a significant change to the Sphinx format, but some new commands do.
In particular, the ratchet we introduce in \S\ref{sec:forward-sec}
for forward security and hybrid post-quantum anonymity introduces a
dramatic amount of state and radically alters the threat model.
Arguably, any formal security arguments for this should employ the 
generalized universal composabiltiy framework \cite{GenUC}, which
requires a new approach to \cite{FormalOnion} that goes beyond the
scope of this article. 

...


