% Xolotl-1-intro

\section{Introduction}

Anonymity systems based on ``onion routing''~\cite{SS03} like Tor or
I2P are known to be vulnerable to correlation attacks by a passive
adversary who can observe both endpoints of a
circuit~\cite{timing-fc2004}, such as a national ISP.  Any attempt to
defeat correlation attacks must take latency into consideration.
% TODO: cite https://blog.torproject.org/blog/one-cell-enough

There are several recent proposals like~\cite{Alpenhorn}
and~\cite{Dissent} that avoid introducing much latency by instead
introducing vast amounts of cover traffic.  There are several issues
with this trade off:  

First, if users can tolerate latency comfortably then they should
use a protocol that does so.  An adversary can extract metadata not
only from timing the protocol but from timing side channels, even
user behavior.  Also there is likely a synergy between anonymity from
latency and cover traffic \cite{??}.

Second, a protocol that scales poorly might favor powerful
organizations with fixed established anonymity sets, such as hiding
which executive at a company asked managers to violate some law.
Instead, we should favor users' whose anonymity needs naturally
cross organizational boundaries, like volunteer organizations,
journalists' sources, whistleblowers, protest organizers, etc.

We favor the opposite trade off in which we accept higher latency but
avoid introducing excessive cover traffic. In effect, we propose to
sacrifice use cases that require low latency like voice, while
offering an inexpensive privacy tool that defeats correlation attacks.
Our target domain includes most text messaging applications and
e-mail, but excludes large file transfers.

\subsection{Messaging API goals} 
% COMMENT: This seems more line API goles than network architecture

A central goal for our network architecture is that messaging is 
{\bf asynchronous} and must work reliably even if sender and receiver
are never online at the same time.  As a result, we ask that messages
can be stored for days to months at nodes in the network.  Generally,
the receiver will select a set of nodes to store his messages and
select a replication level to achieve the desired level of
reliability. While the number of replicas can be disclosed to the
sender, the specific set of replicas is only known to the receiver.
Replicas also do not know of each other, and even collaborating
replicas must not be able to detect that they are storing the same
message for the same receiver.

We assume that the sender initially has a way to securely resolve the
recipient's address to a set of sender-specific single-use reply
blocks (SURBs), for example using an out-of-band exchange or using
name resolution in the GNU Name System~\cite{gns} using a private
label.  However, once a set of SURBs has been established between
sender and receiver, the protocol must maintain the connection
indefinitely, or at least as long as neither sender nor receiver fail
to be online for weeks or months at a time.  Naturally, either side
can also choose to sever the link at any time.
% TODO: SURBs expire too fast for this.  We need semi-public contact
% points.  GNS would leak metadata if used for this.  So I think 
% ``mailbox gatway points'' do this better.

For path selection, sender and receiver require the ability to select
``random'' nodes for routing.  We assume that the network offers a
Byzantine fault-tolerant random peer sampling mechanism, for example
using trusted directory servers~\cite{tordir} or using fault-tolerant
random peer sampling protocols, such as BRAHMS~\cite{brahms}.  Our
design allows peers to in principle impose further constraints on the
path selection, for example limiting the set of guard
nodes~\cite{oneguardisenough}, biasing the selection in favor of
higher bandwidth routers~\cite{findexample} or selecting nodes for
persistent replication based on advertised storage capabilities.
While those choices do matter for privacy, we consider them orthogonal
and thus outside of the scope of this work.
% Discuss: I suspect peer sampling could be biased for an epistemic
% attack on users, so this line sounds like on-going research at
% present.  We do need to sort it out thought so I'm leaving this
% right now.

\subsection{Cryptographic challenges}

We desire anonymity properties that improve on Tor in all respects,
provided our system can achieve similar usage to Tor.  In this vein,
we want cryptographic properties that seems equivalently strong as well.
% I donno if I like this "competitive" framing. 
% Should we mention that Tor might gain post-quantum?

Low latency anonymity tools like Tor achieve {\em forward secrecy}
by employing an ephemeral key exchange on both servers and clients.
We face a cryptographic inconvenience that high latency schemes
like mix networks must ask clients to encrypt to the long term keys
of mix nodes, meaning they lack conventional {\em forward secrecy}.  

There is a superficial similarity between forward secrecy and
post-quantum cryptography: As post-quantum public key primitives
remain young, post-quantum protocols should be analyzed in a hybrid
setting where even ephemeral keys might be compromised.  In other
words, there is a chance that either the classical elliptic curve key
exchange or the post-quantum key exchange might be compromised, but
the chance of both being broken is judged lower.  We therefore wish
to produce a scheme that combines the strengths of both classic and
post-quantum primitives.  However, there are technical obstacles to
deploying a post-quantum key exchange in a mix network, starting with
the simple problem that messages tend to be larger
and computations more expensive.

In this article, we propose Xolotl, a stateful ``ratchet'' based
solution, inspired by the Axolotl ratchet~\cite{TextSecure}, that
extends the Sphinx mix net packet format~\cite{Sphinx}.  
Xolotl provides limited post-quantum protections and forward secrecy
in exchange for leaking some limited correlating information and
increased path length.  

Although harmful, this leakage, and ratchet storage costs, provide an
important parameter for mix network architects:  In any mix network,
mix node key lifetimes correspond with SURB lifetimes, so anytime
contacts do not communicate during a key epoch they must reestablish
connection through a slightly riskier channel.  We believe Xolotl
provides the flexibility needed to extend SURB lifetimes without
making mix node keys too juicy of a target for adversaries.
% George Danezis' fs-mixes~\cite{fs-mix}, or perhaps punctured encryption.

\subsection{Mixing strategy and topology}

We are agnostic to mixing strategy and mix network topology in this
article. 

A priori, SURBs conflict with some approaches to {\em Stop-N-Go mixes}
\cite{StopNGo} in which mixes drop packets that fall outside of client
specified time window.  We expect the network coordination required
for those timing checks sounds unrealistic regardless. 

We do support both packet delays being client controlled as well as
client side delay detection.  In particular, there are no conflicts
between our tweaks and the {\em Poisson mixes} of Loopix \cite{Loopix},
a recent approach to Stop-N-Go mixes.

We introduce several new {\em commands} that Sphinx packets may give
to mix nodes.  These may leak additional information about the
packet's purpose to the node.  We believe a mix network that employs
a stratified topology provides slightly better anonymity than one
employing a free route topology, but the free route weakness might
compound leaks from additional commands.  We therefore recommend a
stratified network topology with these commands restricted to
specific strata.  Again these recommendations agree with Loopix.

