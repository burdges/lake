% Xolotl-3-addresses.tex

\section{Addresses}

...


\subsection{Contact points}\label{subsec:contact_points}

We stated in \S\ref{subsec:crossover} that SURBs might be supplied
by either the sender or the cross over point itself.  We refer to 
a cross over point that supplies the SURB as a {\em contact point}.
We divide contact points according to the issues they address,
as determined by how they authorize the sender,

We observed in \S\ref{subsec:challenges}, and shall discuss more in
\S\ref{sec:forward-sec}, that mix node keys must rotate for forward
secrecy, but that doing so limits SURB lifetimes.  
As a result, clients shall only possess SURBs for contacts with whom
they have corresponded recently.  We cannot even propose that clients
message all their contacts within each key epoch as clients may go
offline for extended periods.

\begin{issue}
How should senders reach recipients who have not messaged them recently?
\end{issue}

Instead, we need contact points that restrict the senders to existing
contacts of the recipient.  Recipients can simply inform all their
senders about these contact points and then keep the supply of SURBs
held by those contact points updated.  These contact points require
authentication credentials to be provided in the command they extract
from $\beta$, so recipients must also communicate the authentication
credentials required by their contact points to their senders.
We note that recipients should still provide senders with SURBs so
that frequent senders do not exhaust the SURBs held by contact points,
as well as for faster communications with frequent contacts.

We could deploy many different authentication credentials ranging 
from single-use tokens as in \cite{agl-pond-hmac}, or
 group signatures~\cite{BBS,VLR}, to a simple shared secret.

\begin{issue}
What credentials should contact points employ to authenticate valid senders?
\end{issue}

We think VLR group signatures \cite{VLR} provide an interesting
option for two reasons.  Firstly, SURBs already provide a single-use
token scheme, so a multi-use scheme complements them nicely.  
Secondly, distinct group member private keys could be printed onto
ordinary business cards in advance and even the initial message
identifies the card carrying that member private key.  Identities
can be assigned to these keys after communication is established.  
Importantly senders can be revoked without invalidating all existing
keys.  We expect VLR group signatures incur notable computational
overhead for the contact point.  % Worse, this cost grows as more
% senders get revoked and revokations are surprisingly frequent in
% practice \cite{agl-pond-hmac}.
% TODO: Verify that VLR groups signatures scale badly in revokations

As a rule, contact points are an extremely low bandwidth channel
since their supply of SURBs can easily be exhausted.  We must not
reveal any identifying information about the sender to the contact 
point itself during authentication, but authentication should ideally
reveal some $\iota$ to the contact point from which the recipient
can identify the sender.  VLR group signatures and single-use tokens
provide this $\iota$, but a simple shared secret fails to.  We must
now communicate $\iota$ from the contact point to the recipient. 
We could encode $\iota$ into the SURB log, or even empty space in
$\delta$, but this poses two minor problems: 
 Another hop could corrupt $\iota$ masking the true sender.
 Also $\iota$ is large for VLR group signatures.  
%TODO: Verify that $\iota$ is large for VLR group signatures.
We could accumulate a log of $\iota$s on the contact point and send
this log to the recipient when few SURBs remain, but this only
provides verification after the fact.

%TODO: Talk about if we're implementing VLR group signatures?
%TODO: Explain why we cannot simply move the contact points under DoS.


\subsection{New contacts}\label{subsec:greeting_points}

% \begin{issue}
% What about new contacts?
% \end{issue}
% TODO: This seems kinda silly?!?

We want special contact points called {\em greeting points} that
do not restrict senders to existing contacts.  A priori, these sound
vulnerable to denial-of-service attacks, and might suffer from SPAM.
We hope that simply moving the advertised greeting points occasionally
works, so long as existing contacts never need them thanks to other
contact points.

We need a facility for users to introduce other users to one another.
because doing so seems essential for group communication.  If Alice 
knows both Bob and Carol then Alice can introduce Bob and Carol by
providing each with either a SURB if the introduction should happen
in a timely fashion, or with authentication credentials for each others'
contact points.  

\begin{issue}
How do users introduce one another for group messaging?
\end{issue}

We require that Alice does not compromise her ability to contact
either Bob or Carol by doing the introduction and that Bob and Carol
can each distinguish the others' messages' from Alice. 

We observe that introductions cannot only work by sharing SURBs since
contacts do not always posses SURBs for one another.  Introductions
that share SURBs must also account for the faster SURB depletion rate.
We shall therefore focus on schemes that merely supply authentication
credentials for contact points.

In contact points use single-use tokens, then Alice can easily share
single-use tokens from Bob and Carol of course.  These tokens could
hint that an introduction will take place.

If contact points use a VLR scheme, then Alice creates the delivery
authorization signature that Bob and Carol must use.  In this case,
we could employ a second VLR identity, or some tweak facility, to
provide a hint about the message being an introduction. 

We recall that $\delta$ must be encrypted at the cross over point.
In principle, Alice could supply Bob and Carol with the bottom layer
encryption and authentication keys for $\delta$, as well as the
initial root key state for their new Axolotl ratchet.  % \S\ref{subsec:crossover}
In this scheme, both Bob and Carol must receive the introduction
from Alice though.  Instead, we suggest that $\delta$ itself employ
a public key encryption scheme.

% Axolotl-SKPQ can still supply ...


% TODO: How much do we wish to explain here? 
% TODO: Our TOFU protocol even?  \cite{tofu} 
 


